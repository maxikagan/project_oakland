\section{Methodological Appendix: Constructing Visitor Partisan Lean}
\label{sec:appendix_methodology}

\subsection{Overview}

This appendix describes the methodology for constructing measures of visitor partisan lean for points of interest (POIs) across the United States. We link anonymized mobile phone location data to Census Block Group (CBG)-level presidential election results to estimate the political composition of visitors to retail establishments, restaurants, and other commercial locations.

The final output is a panel dataset at the POI-month level containing weighted average Republican vote share of visitors, computed using both 2016 and 2020 presidential election results.

\subsection{Data Sources}

\subsubsection{Foot Traffic Data: Advan Monthly Patterns}

We use Advan (formerly SafeGraph) Monthly Patterns data covering January 2019 through July 2025 (79 months) for all 50 U.S. states plus the District of Columbia. The raw data consists of 2,096 compressed CSV files totaling approximately 804 GB.

For each POI-month observation, the dataset includes a unique identifier (\texttt{PLACEKEY}), business attributes (brand, category, NAICS code), location information (city, state, Census Block Group), and critically, a JSON-encoded field \texttt{VISITOR\_HOME\_CBGS} containing a dictionary mapping visitor home CBGs to visitor counts. For example:

\begin{verbatim}
{"060371234001": 45, "060371234002": 23, "060371235001": 12}
\end{verbatim}

This indicates that 45 visitors came from CBG 060371234001, 23 from CBG 060371234002, and 12 from CBG 060371235001.

Advan applies differential privacy protections: CBGs with fewer than 4 visitors are suppressed, and visitor counts are subject to noise injection. These protections may introduce measurement error but do not systematically bias partisan lean estimates.

\subsubsection{Election Data: CBG-Level Presidential Vote Estimates}

We use CBG-level presidential election results from the ``Main Method'' approach with RLCR (Registered Voter List with Candidate Records) methodology. The data provides estimated vote counts at the Census Block Group level for the 2016 and 2020 presidential elections, covering all 283,900 CBGs in the contiguous United States.

The RLCR method estimates block group-level vote shares by combining precinct-level official election returns with voter file data that includes geocoded addresses and modeled partisanship. This produces more granular estimates than precinct-level data alone, though estimates are subject to modeling uncertainty.

\subsubsection{Metropolitan Statistical Area Crosswalk}

We use the NBER CBSA-to-FIPS County Crosswalk (2023 delineations) to map each POI to its Metropolitan Statistical Area based on the county portion of the POI's CBG code.

\subsection{Data Processing Pipeline}

\subsubsection{Step 1: Construct National CBG Partisan Lean Lookup}

We create a national lookup table containing Republican two-party vote share for each Census Block Group:

\begin{equation}
\text{two\_party\_rep\_share}_{c,t} = \frac{\text{Republican votes}_{c,t}}{\text{Republican votes}_{c,t} + \text{Democratic votes}_{c,t}}
\end{equation}

where $c$ indexes CBGs and $t \in \{2016, 2020\}$ indexes election years. For CBGs with zero total votes, we assign a neutral value of 0.5. The final lookup table contains 283,900 CBGs with partisan lean measures for both election years.

\subsubsection{Step 2: Single-Pass Partisan Lean Computation}

We compute visitor partisan lean using an efficient single-pass architecture. Rather than filtering the raw data by state first (which would require reading each of 2,096 files 51 times for a total of 107,000 file reads), we process each source file exactly once, computing partisan lean for all POIs in that file regardless of state. This reduces total file reads from 107,000 to 2,096---a 52-fold improvement in I/O efficiency.

The processing is implemented as a SLURM array job with 2,096 tasks, one per source file. Each task:

\begin{enumerate}
    \item Loads the national CBG partisan lean lookup into memory as Python dictionaries for $O(1)$ lookup performance
    \item Reads one compressed CSV file, selecting only required columns
    \item For each POI-month observation, parses the \texttt{VISITOR\_HOME\_CBGS} JSON and computes weighted partisan lean
    \item Outputs results to a Parquet file
\end{enumerate}

\subsubsection{Partisan Lean Computation}

For each POI-month observation, we compute the visitor-weighted average Republican vote share:

\begin{equation}
\text{rep\_lean}_{i,m,t} = \frac{\sum_{c \in C_i} \left( \text{rep\_share}_{c,t} \times \text{visitors}_{c,i,m} \right)}{\sum_{c \in C_i} \text{visitors}_{c,i,m}}
\end{equation}

where $i$ indexes POIs, $m$ indexes months, $t$ indexes election years, $C_i$ is the set of visitor home CBGs for POI $i$ that match the election data lookup, and $\text{visitors}_{c,i,m}$ is the count of visitors from CBG $c$ to POI $i$ in month $m$.

Importantly, because we use a \textit{national} CBG lookup table, cross-state visitors are handled correctly. A POI in California with visitors from Nevada will correctly match those Nevada CBGs to their partisan lean values.

We track the match rate for each observation:

\begin{equation}
\text{pct\_matched}_{i,m} = \frac{\text{matched visitors}_{i,m}}{\text{total visitors}_{i,m}} \times 100
\end{equation}

In practice, we achieve match rates of approximately 99\%, with unmatched CBGs typically arising from boundary changes between census years or CBGs in Alaska and Hawaii (which are not in the contiguous USA election data).

\subsubsection{Step 3: Combine and Partition Outputs}

After all 2,096 file-level jobs complete, we combine the outputs into a single national dataset partitioned by year-month for efficient downstream analysis. Summary diagnostics are generated including state-level and brand-level aggregations.

\subsection{Key Methodological Decisions}

\subsubsection{Two-Party Vote Share}

We compute Republican vote share as a proportion of the two-party (Republican + Democratic) vote rather than total votes. This approach excludes third-party votes, creates a bounded measure in $[0, 1]$, and is standard in political science literature.

\subsubsection{Visitor-Count Weighting}

Partisan lean is computed as a visitor-count-weighted average rather than a simple average across CBGs. This ensures that CBGs contributing more visitors have proportionally greater influence on the final measure.

\subsubsection{Both Election Years}

We compute partisan lean using both 2016 and 2020 election data for all POI-months, regardless of the observation date. This allows researchers to assess sensitivity to election year choice and provides robustness checks.

\subsubsection{Handling Unmatched CBGs}

Some visitor home CBGs cannot be matched to the election data lookup due to boundary changes or geographic coverage. We exclude unmatched visitors from the weighted average and track the match rate for each observation, enabling researchers to filter observations with low match rates if desired.

\subsection{Output Data}

The primary output is a POI-month panel with the following key variables:

\begin{itemize}
    \item \texttt{placekey}: Unique POI identifier
    \item \texttt{date\_range\_start}: First day of month
    \item \texttt{brand}: Brand name (null for non-chain POIs)
    \item \texttt{region}: Two-letter state code
    \item \texttt{cbsa\_title}: Metropolitan Statistical Area name
    \item \texttt{rep\_lean\_2020}: Weighted Republican vote share (2020 election), range $[0, 1]$
    \item \texttt{rep\_lean\_2016}: Weighted Republican vote share (2016 election), range $[0, 1]$
    \item \texttt{total\_visitors}: Total visitor count from all CBGs
    \item \texttt{matched\_visitors}: Visitor count from CBGs with election data
    \item \texttt{pct\_visitors\_matched}: Match rate percentage
\end{itemize}

\subsection{Computational Implementation}

The pipeline is implemented on the UC Berkeley Savio HPC cluster using SLURM for job scheduling. The single-pass architecture processes 2,096 source files in parallel, with each task completing in approximately 45 seconds. Total wall-clock time for the full pipeline is approximately 1-2 hours with 50 concurrent tasks.

\subsection{Limitations}

\begin{enumerate}
    \item \textbf{Mobile phone sampling:} The Advan data may not be representative of all visitors due to differential smartphone adoption rates.
    \item \textbf{Privacy-preserving noise:} Differential privacy protections may attenuate measured partisan differences.
    \item \textbf{Ecological inference:} We observe aggregate CBG-level voting patterns, not individual preferences.
    \item \textbf{Home location inference:} The ``home'' CBG is inferred from nighttime location patterns and may not reflect current residence for all individuals.
\end{enumerate}
