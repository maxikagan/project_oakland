\section{Conclusion}\label{sec:conclusion}

Firms today operate in an environment of intense political polarization. Their stakeholders---employees, customers, investors, communities---hold increasingly divergent political views, creating challenges that traditional stakeholder theory did not anticipate. This paper introduces stakeholder political pluralism as a framework for understanding and measuring political heterogeneity within and across stakeholder groups.

Using novel data linking mobile phone location records to election results and employment histories to voter registration, we develop comprehensive measures of consumer and employee political composition for millions of U.S. business establishments. Our analysis reveals substantial variation in stakeholder political composition across geography, industry, and brand. While location explains most of this variation, meaningful differences exist among businesses serving the same neighborhoods. Some brands attract systematically more Republican or Democratic visitors than their geographic footprints would predict; employees and customers are correlated but imperfectly aligned.

These descriptive patterns establish a foundation for understanding political sorting in commercial activity. But the implications extend beyond measurement. Our performance analysis suggests that stakeholder political pluralism matters for firm outcomes, particularly during periods of elevated partisan conflict. When employees and customers diverge politically, and when political issues are salient, misalignment is associated with weaker economic performance. The caught-in-the-middle problem is not merely theoretical; it manifests in measurable outcomes.

For managers, our findings underscore the importance of understanding stakeholder political composition before engaging in political positioning. For a firm with aligned stakeholders, political engagement may be relatively straightforward---the risks are lower, the potential rewards clearer. For pluralistic firms, political engagement requires greater caution. Any position will alienate some stakeholders, and the costs of misalignment intensify during politically charged periods.

More broadly, our work speaks to the challenge of stakeholder management in divided societies. Stakeholder theory's optimistic vision---that firms can create value for all stakeholders simultaneously---assumes that stakeholder interests can be reconciled. When stakeholders disagree fundamentally on political questions, this reconciliation becomes more difficult. Understanding where pluralism exists, and how it affects firm outcomes, is a first step toward developing strategies for navigating these tensions.

We close by noting that political polarization shows few signs of abating. If anything, the integration of politics into consumer identity and workplace dynamics appears to be deepening. The stakeholder political pluralism challenge will likely become more acute, not less. Developing frameworks for understanding and managing this challenge is an important task for management research and practice alike.
