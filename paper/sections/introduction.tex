\section{Introduction}\label{sec:introduction}

In April 2023, Bud Light partnered with transgender influencer Dylan Mulvaney for a social media promotion. The backlash was swift and severe: conservative consumers organized boycotts, sales plummeted by 26\%, and the brand lost its position as America's best-selling beer. Yet the company's decision to distance itself from Mulvaney triggered an equally forceful response from progressive consumers and LGBTQ+ advocacy groups, who accused Anheuser-Busch of abandoning its values under pressure. Caught between stakeholder groups with irreconcilable preferences, the company watched its market capitalization decline by billions of dollars \citep{marketing2023bud}.

This episode---and others like it involving Disney, Target, Nike, and Chick-fil-A---illustrates a fundamental challenge facing firms in an era of political polarization: how do organizations manage stakeholders when those stakeholders fundamentally disagree? Stakeholder theory, the dominant paradigm for understanding firm-stakeholder relationships, has traditionally emphasized the possibility of creating value for multiple stakeholders simultaneously \citep{freeman1984strategic, jones1995instrumental}. But this optimistic view assumes that stakeholder interests can be aligned, or at least that trade-offs can be navigated without alienating critical constituencies. When stakeholders hold opposing political views---views tied to identity, values, and social group membership---the possibility of alignment becomes uncertain.

Despite the growing salience of this challenge, we lack systematic evidence about how stakeholder political composition varies across firms. Do some firms serve politically homogeneous customer bases while others navigate divided markets? Are employees and customers of the same firm politically similar, or do they diverge? And when stakeholder groups are misaligned, does it affect firm performance? Answering these questions requires measuring the political composition of firm stakeholders at scale---a task that has been infeasible until recently due to data limitations.

This paper develops novel measures of stakeholder political ideology and uses them to examine ideological alignment across stakeholder groups. We construct measures of \textit{consumer partisan composition} by linking granular foot traffic data to election results: for millions of retail establishments, restaurants, and service locations, we observe which Census Block Groups (CBGs) visitors come from and assign partisan lean based on those neighborhoods' presidential voting patterns. We construct measures of \textit{employee partisan composition} by linking employment records to voter registration data, enabling us to characterize the political orientation of each firm's workforce. Together, these measures provide an unprecedented view of stakeholder ideology across the American economy.

Our empirical approach proceeds in several stages. First, we document descriptive patterns in consumer partisan composition across industries, geographies, and brands. The data reveal substantial variation: while location explains the majority of variance in visitor partisan lean---businesses in Republican areas tend to attract Republican visitors---meaningful differences exist even among businesses in the same neighborhood. Category matters: gun shops and agricultural suppliers attract more conservative visitors than organic grocers and yoga studios, even controlling for location. Brand matters too: Walmart and Target differ in visitor partisan composition beyond what their geographic footprints would predict.

Second, we examine the relationship between employee and consumer political composition. Using a novel entity resolution procedure to link foot traffic data to employment records, we analyze whether the political orientation of a firm's workforce correlates with that of its customers. This analysis speaks directly to stakeholder theory's assumptions about alignment: if employees and customers are systematically similar politically, firms may face less pressure from cross-stakeholder tensions. If they diverge, the challenge of managing conflicting stakeholder preferences becomes more acute.

Third, we examine performance implications. Using SafeGraph Spend data on store-level revenue and exploiting temporal variation in political conflict salience (the Partisan Conflict Index from the Philadelphia Fed), we test whether stakeholder ideological alignment affects firm outcomes. Our identification strategy relies on the intuition that misalignment should matter more when political issues are salient---when the partisan ``temperature'' is high. If stakeholder ideological composition has no effect on firm performance, we would expect the coefficient on misalignment to be zero regardless of political climate. If it matters, we should observe that misalignment hurts performance more during periods of elevated partisan conflict.

Our analysis makes four contributions. First, we develop new measures of stakeholder political composition that can be applied broadly across the economy. Prior research has examined consumer ideology for small samples of firms using surveys \citep{panagopoulos2017all} or inferred preferences from social media \citep{schoenmueller2023polarized}, but no existing measures cover the breadth of commercial activity we observe. Our foot traffic-based approach generates partisan composition estimates for millions of establishments, enabling systematic analysis previously impossible.

Second, we provide the first large-scale empirical documentation of how stakeholder ideology varies across firms. We characterize the distribution of consumer partisan lean across industries, identify which brand categories show the most ideological differentiation, and quantify how much of the variation is explained by location versus business characteristics. These descriptive facts establish a foundation for understanding political sorting in consumer markets.

Third, we examine the relationship between employee and consumer ideology---a question central to stakeholder theory but never empirically addressed at scale. The correlation (or lack thereof) between these stakeholder groups has direct implications for how firms should think about political positioning. Our evidence speaks to whether the ``caught in the middle'' problem is widespread or confined to specific firm types.

Fourth, we provide evidence on the performance consequences of stakeholder ideological alignment. While much research has examined how firms' political \textit{activities}---donations, lobbying, CEO statements---affect outcomes \citep{hillman2004political, hadani2017institutional}, less is known about how stakeholder \textit{composition} influences performance. Our analysis begins to fill this gap, testing whether firms with more aligned stakeholders outperform those navigating divided constituencies.

These contributions extend stakeholder theory in several ways. Most importantly, we introduce ideological heterogeneity as a dimension of stakeholder management. The classic stakeholder framework identifies groups---customers, employees, suppliers, communities---and considers their interests \citep{freeman1984strategic}. But within each group, members may disagree profoundly. A firm's customers are not a monolith; nor are its employees. Understanding \textit{within-stakeholder} diversity, and \textit{across-stakeholder} alignment, enriches the stakeholder perspective and provides practical guidance for managers facing polarized environments.

The paper proceeds as follows. Section~\ref{sec:theory} develops our theoretical framework, engaging with stakeholder theory, the corporate political activity literature, and research on political consumerism. Section~\ref{sec:data} describes our data sources and measure construction. Section~\ref{sec:results} presents empirical findings on descriptive patterns, stakeholder alignment, and performance implications. Section~\ref{sec:discussion} discusses theoretical contributions and limitations. Section~\ref{sec:conclusion} concludes.
