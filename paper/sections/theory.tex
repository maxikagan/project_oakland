\section{Theory}\label{sec:theory}

We develop a theoretical framework grounded in stakeholder theory that incorporates political ideology as a critical dimension of stakeholder relationships. Our framework engages with three streams of literature: stakeholder theory and its assumptions about alignment, corporate political activity and stakeholder reactions, and research on political consumerism and employee ideology. We then develop predictions about how stakeholder ideological composition varies across firms and how alignment affects firm outcomes.

Our theoretical development speaks to what \citet{bridoux2023new} call the ``new stakeholder theory on organizational purpose,'' which grapples with questions of stakeholder enfranchisement and value distribution. By introducing ideological composition as a dimension of stakeholder relationships, we contribute to understanding how stakeholder preferences---and conflicts among them---shape organizational strategy.

\subsection{Stakeholder Theory and the Alignment Problem}

Stakeholder theory posits that firms should create value for all stakeholders, not just shareholders \citep{freeman1984strategic}. This perspective recognizes that employees, customers, suppliers, communities, and other groups affect and are affected by firm activities, and that successful organizations manage these relationships to generate mutual benefit. The instrumental version of stakeholder theory further argues that attending to stakeholder interests is not merely ethical but also strategically advantageous: firms that treat stakeholders well enjoy better long-term performance \citep{jones1995instrumental}.

Central to this framework is the stakeholder salience model developed by \citet{mitchell1997toward}. Salience---the degree to which managers prioritize competing stakeholder claims---depends on three attributes: power (the stakeholder's ability to impose its will), legitimacy (the social appropriateness of the stakeholder's claim), and urgency (the time-sensitivity and criticality of the claim). Stakeholders possessing all three attributes demand immediate managerial attention; those with fewer attributes may be ignored or deprioritized. This model explains why managers attend to some stakeholders more than others, but it assumes that stakeholder preferences are relatively stable and knowable.

Recent work has advanced what \citet{mcgahan2021integrating} calls the ``new stakeholder theory'' (NST), which brings greater precision to understanding stakeholder involvement in organizations. The NST grapples with two canonical questions: which stakeholders are enfranchised in organizations, and how is value created through stakeholder collaboration distributed? \citet{bridoux2022stakeholder} trace how stakeholder theory has evolved from normative arguments about stakeholder consideration to formal analysis of how stakeholder composition affects strategic outcomes. A central insight is that stakeholders bind resources to organizations---employees contribute human capital, customers provide revenue, suppliers offer inputs---and understanding these resource contributions illuminates why stakeholder management is strategically important \citep{barney2018stakeholder}.

An implicit assumption in much stakeholder theory is that stakeholder interests, while diverse, can be reconciled or balanced. \citet{freeman2010stakeholder} argue that value creation need not be zero-sum: clever managers can find solutions that benefit multiple stakeholders simultaneously. This ``both/and'' perspective underpins stakeholder capitalism's normative appeal. Yet it sidesteps a difficult question: what happens when stakeholder interests are not merely different in magnitude but opposite in direction?

Political ideology presents precisely this challenge. When a firm's customers include both liberals and conservatives, satisfying one group's preferences may alienate the other. This is not a problem of incomplete information or misaligned incentives that can be solved through better contracts or communication. It is a fundamental conflict rooted in different values, worldviews, and group identities. The stakeholder framework, developed before polarization reached current levels, does not fully address this scenario.

Our research extends stakeholder theory by introducing ideological heterogeneity as a dimension of stakeholder composition. We examine not only whether firms have customers and employees but what those stakeholders believe politically---and whether different stakeholder groups share similar political orientations. This extension enables us to address the ``caught in the middle'' problem that firms increasingly face.

\subsection{Political Ideology as a Stakeholder Attribute}

Political ideology---an individual's system of beliefs about how society should be organized---shapes economic behavior in multiple ways. A large literature documents that partisan identity influences consumption patterns, employment choices, and reactions to firm activities. These effects operate through several mechanisms.

\textit{Value expression.} Consumers and employees derive utility from associating with organizations that share their values \citep{bhattacharya2003consumer}. Just as consumers may prefer products from firms with strong environmental commitments, they may prefer firms whose political positioning aligns with their own. The rise of ``political consumerism''---using market choices to express political preferences---reflects this tendency \citep{stolle2005consumers, endres2017resisting}. When consumers boycott firms over political stances or buycott firms they wish to support, they are expressing identity through market behavior.

\textit{In-group favoritism.} Social identity theory suggests that individuals favor members of their own groups and disfavor out-groups \citep{tajfel1979integrative}. When partisanship becomes a social identity---as it increasingly has in the United States \citep{mason2018one, iyengar2019origins}---consumers and employees may prefer organizations perceived as co-partisan. This extends beyond explicit political stances: even implicit cues about a firm's political orientation (its customer base, employee composition, geographic footprint) may trigger in-group preferences.

\textit{Trust and legitimacy.} Ideology shapes perceptions of organizational legitimacy. Conservative individuals may view firms with progressive workforces as less trustworthy; liberals may distrust firms associated with conservative causes. Because legitimacy affects willingness to transact---customers must believe the firm will fulfill its promises; employees must believe the firm will treat them fairly---ideological mismatch can erode the foundation of stakeholder relationships.

These mechanisms suggest that stakeholder political composition is not merely a demographic curiosity but a strategically relevant attribute. Firms serving politically homogeneous stakeholder bases may find it easier to maintain trust and satisfy expectations. Firms serving diverse bases face the challenge of appealing to groups with conflicting preferences.

\subsection{Corporate Political Activity and Stakeholder Reactions}

A substantial literature examines corporate political activity (CPA)---the actions firms take to influence government policy---and its consequences \citep{hillman2004political}. CPA includes lobbying, political donations, direct engagement with policymakers, and public advocacy on political issues. Research has documented both benefits (favorable policy outcomes, regulatory access) and costs (stakeholder backlash, reputational damage) of political engagement \citep{hadani2017institutional}.

Recent work has increasingly focused on how stakeholders respond to corporate political involvement. \citet{burbano2021effect} finds that employees respond negatively to CEO political donations that conflict with their own preferences, with effects on job satisfaction and turnover intentions. \citet{hambrick2019political} documents that CEO political contributions affect customer perceptions and firm reputation. \citet{melloni2019corporate} shows that firms face pressure from employees and customers when they engage with political issues, and that responses depend on the political composition of these stakeholders.

One stream of this literature examines ``CEO activism''---public statements by corporate leaders on political and social issues \citep{chatterji2019origins}. While CEO activism can enhance brand loyalty among sympathetic customers and increase employee engagement among like-minded workers, it risks alienating stakeholders on the other side. \citet{korschun2022taking} theorizes that CEO activism represents a risky bet: the firm gains among stakeholders who share the CEO's position but loses among those who disagree.

This research highlights a critical point: the consequences of corporate political engagement depend on stakeholder composition. A firm with a homogeneously liberal customer base may benefit from progressive CEO activism; the same activism could devastate a firm serving conservatives. Recent work by \citet{mckean2024ideologies} examines this directly, finding that ideological alignment between a firm's upper echelons and its stakeholders influences participation in progressive corporate activism. Similarly, \citet{gupta2025political} theorize how growing political polarization reshapes firms' nonmarket strategies over the policy life cycle, showing that polarization delays social consensus and intensifies both activism and industry resistance. Yet existing research typically treats stakeholder composition as unobserved, inferring it indirectly from reactions or modeling it theoretically. We contribute by directly measuring stakeholder ideology, enabling precise analysis of how composition moderates responses to political activity.

\subsection{The Employee Dimension}

While much research on stakeholder politics focuses on consumers, employees represent an equally important stakeholder group. Employees' political beliefs shape their workplace experiences, organizational commitment, and willingness to engage in discretionary effort.

Value congruence---the alignment between individual and organizational values---is a well-established predictor of employee outcomes \citep{kristof1996person, edwards2009person}. When employees perceive that their employer shares their values, they report higher job satisfaction, stronger organizational identification, and lower turnover intentions. Extending this to political values, we would expect employees whose political orientation aligns with their employer's (or their coworkers') to experience greater fit and commitment.

Recent work has begun to examine political dynamics in the workplace directly. \citet{sisco2022managing} finds that political diversity in workplaces can generate conflict, with effects on team performance and employee well-being. \citet{burbano2021effect} documents that employees care about their employers' political activities and may exit when those activities conflict with their beliefs. The Politics at Work project \citep{mcconnell2022economic} reveals substantial variation in employee partisan composition across firms and shows that this composition predicts corporate political behavior.

An important question is whether employee and customer political composition covary. If they do, firms may face reinforcing pressures from aligned stakeholder groups. If they diverge, firms must navigate cross-stakeholder tensions---progressive employees serving conservative customers, or vice versa. This divergence could create internal conflict, affect customer service, and complicate corporate political positioning. We provide the first large-scale evidence on the correlation between employee and customer ideology.

\subsection{Within-Stakeholder Diversity}

Beyond cross-stakeholder alignment, we consider within-stakeholder diversity. A firm's customers are not a homogeneous group; they may themselves hold diverse political views. Similarly, a firm's employees may span the political spectrum. The degree of within-group heterogeneity has implications for stakeholder management.

Consider two firms with the same mean customer partisan lean of 50\% Republican. Firm A's customers are uniformly distributed: half are strong Republicans, half are strong Democrats. Firm B's customers cluster near the center: most are moderates or weak partisans. Though both firms have the same average, their strategic situations differ dramatically. Firm A faces the ``caught in the middle'' problem acutely: any political move will alienate half its customers. Firm B, serving a politically moderate base, may have more latitude.

This insight connects to research on market positioning under heterogeneous preferences \citep{caves1977industrial}. Firms may differentiate to serve distinct segments, pursue broad appeal strategies, or attempt niche positioning. When heterogeneity is political, differentiation becomes more fraught: unlike product preferences, political preferences are tied to identity and group membership. Attempting to serve ``both sides'' may be perceived as inauthentic or unprincipled, pleasing neither group.

We therefore examine not only mean stakeholder ideology but its dispersion. Firms with more politically diverse customer bases face different challenges than those serving homogeneous markets, and this diversity may moderate the relationship between ideological alignment and performance.

\subsection{Temporal Variation: The Role of Political Climate}

The salience of political ideology as a stakeholder attribute varies over time. During periods of elevated political conflict---contested elections, Supreme Court decisions, highly publicized political events---partisan identity becomes more accessible and influential. During quieter periods, other considerations may dominate consumer and employee behavior.

This temporal variation provides an identification opportunity. If stakeholder ideological composition affects firm performance, we would expect this effect to be stronger when political issues are salient. When the ``partisan temperature'' is high, consumers are more likely to factor politics into purchasing decisions, employees are more likely to evaluate employers through a political lens, and misalignment should matter more. When the temperature is low, ideology recedes in importance, and alignment or misalignment should have weaker effects.

We exploit this intuition using the Partisan Conflict Index (PCI), a monthly measure of political disagreement developed by the Federal Reserve Bank of Philadelphia \citep{azzimonti2018partisan}. The PCI captures the intensity of partisan conflict based on news coverage of political disagreements. By interacting stakeholder ideological alignment with the PCI, we can test whether alignment matters more during periods of heightened political salience.

This approach addresses endogeneity concerns that plague cross-sectional comparisons. One might worry that firms with more aligned stakeholders differ from misaligned firms in unobservable ways that also affect performance. But if the alignment-performance relationship varies with political climate---stronger when PCI is high, weaker when PCI is low---this suggests the relationship is driven by political mechanisms rather than confounds. The interaction provides quasi-experimental leverage on a fundamentally correlational question.

\subsection{Summary of Theoretical Predictions}

Our theoretical framework generates several predictions, which we organize by the type of analysis required to test them.

First, we expect substantial variation in stakeholder ideological composition across firms. While geography will explain much of this variation---businesses in Republican areas attract Republican visitors---we predict that industry, brand, and business type will also matter. Certain categories (e.g., organic grocers, gun shops) should attract politically distinct customer bases even within the same geographic area.

Second, we expect employee and consumer political composition to be positively correlated but imperfectly aligned. Firms in conservative areas will tend to have both conservative employees and conservative customers, generating positive correlation. But the correlation should be imperfect because employees and customers are selected through different processes: employees choose employers based on job availability, compensation, and career opportunities; customers choose businesses based on product offerings, convenience, and price. These distinct selection processes should introduce divergence.

Third, we expect stakeholder ideological alignment to be associated with firm performance, particularly during periods of elevated political conflict. When employees and customers share political orientations, the firm faces less tension in its stakeholder management and can more easily maintain trust across groups. When they diverge, the firm must navigate conflicting expectations, risking backlash from one group or the other. This relationship should intensify when political issues are salient, as captured by high PCI.

Fourth, we expect within-stakeholder diversity to moderate these relationships. Firms serving more politically homogeneous customer bases should show stronger alignment effects because there is less diversity to moderate reactions. Firms serving diverse bases may see dampened effects as opposing reactions cancel out.

These predictions guide our empirical analysis. We turn next to describing the data and methods used to test them.
