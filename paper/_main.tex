\documentclass[10pt, letterpaper]{article}

% ===================== CAPTIONS ==========================
\usepackage{caption}
\usepackage{subcaption}
\usepackage{threeparttable}

% ===================== FLOAT SETTINGS ====================
\renewcommand{\topfraction}{0.9}
\renewcommand{\bottomfraction}{0.9}
\setcounter{topnumber}{3}
\setcounter{bottomnumber}{3}
\setcounter{totalnumber}{6}
\setcounter{dbltopnumber}{3}
\renewcommand{\dbltopfraction}{0.9}
\renewcommand{\textfraction}{0.05}
\renewcommand{\floatpagefraction}{0.5}
\renewcommand{\dblfloatpagefraction}{0.5}

% ===================== FIGURES ===========================
\usepackage{float}
\usepackage{pgfplots}
\usepackage{mwe}
\pgfplotsset{compat=1.16}
\usepackage{placeins}

% ========================= FONTS =========================
\usepackage{libertine}
\usepackage[libertine]{newtxmath}

% ===================== FOOTNOTES =========================
\usepackage[hang,flushmargin]{footmisc}

% ====================== GRAPHICS =========================
\usepackage{graphicx}
\usepackage{adjustbox}
\usepackage{lscape}
\usepackage{pdflscape}
\usepackage{rotating}
\usepackage{xcolor}

% ===================== HYPERLINKS ========================
\usepackage[hidelinks]{hyperref}
\usepackage{url}
\urlstyle{sf}

% ==================== MATH AND SYMBOLS ===================
\let\Bbbk\relax
\usepackage{amsmath}
\usepackage{amssymb}
\usepackage{bbm}
\usepackage{mathtools}
\usepackage{siunitx}

% ==================== OTHER FEATURES =====================
\usepackage{authblk}
\usepackage{atbegshi,picture}
\usepackage{markdown}
\usepackage{tocbibind}
\usepackage{minitoc}

% ====================== PAGE LAYOUT ======================
\usepackage[margin=1in]{geometry}
\usepackage{setspace}
\usepackage{changepage}

% ===================== REFERENCES ========================
\usepackage{natbib}
\setcitestyle{round,authoryear}
\usepackage{bibunits}

% ====================== TABLES ===========================
\usepackage{array}
\usepackage{booktabs}
\usepackage{makecell}
\usepackage{ragged2e}
\usepackage{multirow}
\usepackage{tabularray}
\usepackage{longtable}
\usepackage{tabu}

% ====================== TEXT =============================
\usepackage{csquotes}
\usepackage{indentfirst}
\usepackage{verbatim}
\usepackage{xifthen}
\usepackage{enumitem}
\usepackage{lipsum}

% ====================== BIBLIOGRAPHY =====================
\bibpunct[, ]{(}{)}{;}{a}{}{,}
\def\bibhang{24pt}
\def\newblock{\ }
\def\BIBand{and}
\def\bibfont{\small}
\def\bibsep{\smallskipamount}
\setcitestyle{notesep={:}}

% ===================== NEW COMMANDS ======================
\newcommand{\nocontentsline}[3]{}
\newcommand{\tocless}[2]{\bgroup\let\addcontentsline=\nocontentsline#1{#2}\egroup}
\newcommand{\removeperiod}{\@ifnextchar.{\@gobble}\relax}
\newcommand\cites[1]{\citeauthor{#1}'s\ (\citeyear{#1})}

% Placeholder command for figures
\newcommand{\placeholderfigure}[3]{%
    \begin{figure}[htbp]
        \centering
        \fcolorbox{black}{gray!20}{%
            \parbox{0.85\textwidth}{%
                \vspace{3cm}
                \centering
                \textbf{#1}\\[0.5em]
                \textit{#2}
                \vspace{3cm}
            }%
        }
        \caption{#3}
    \end{figure}
}

% Placeholder command for tables
\newcommand{\placeholdertable}[2]{%
    \begin{table}[htbp]
        \centering
        \caption{#1}
        #2
    \end{table}
}

% ---------------------------------------------------------------------------------

\begin{document}

\title{Stakeholder Political Pluralism: A Stakeholder Theory Perspective on Employee and Consumer Divergence}
% Alternative titles:
% - Measuring Stakeholder Ideology: Political Alignment Across Employees and Consumers
% - Stakeholder Theory and Political Pluralism: Evidence from Employees and Consumers
% - When Stakeholders Disagree: Political Heterogeneity and Firm Performance

\author{Max Kagan}
\affil{Columbia Business School}

\date{\today}


\begin{singlespacing}
\maketitle

\begin{abstract}
\noindent
Stakeholder theory urges firms to attend to the interests of all stakeholders but offers limited guidance when stakeholder groups hold fundamentally divergent preferences. This paper introduces the concept of stakeholder political pluralism---the extent to which a firm's stakeholder groups differ in their political orientations---and examines its implications for stakeholder management. Using 596 million point-of-interest-month observations from mobile location data linked to election results and 45 million employment records linked to voter registration, I construct novel measures of political composition for two critical stakeholder groups: consumers and employees. I document substantial variation in stakeholder political pluralism across firms, industries, and geographies. While employee and consumer political composition are positively correlated, alignment is imperfect---many firms navigate stakeholder groups with divergent political orientations. I find that pluralism is associated with weaker firm performance, particularly during periods of elevated partisan conflict. These findings contribute to stakeholder theory by introducing political heterogeneity as a dimension of stakeholder relationships and by illuminating the challenges firms face when stakeholder alignment is unattainable.

\bigskip
\noindent\textbf{Keywords:} Stakeholder theory, political ideology, consumer behavior, employee ideology, corporate political activity
\end{abstract}

\end{singlespacing}
\thispagestyle{empty}
\pagebreak

\clearpage
\doublespacing


% INTRODUCTION --------------------------------------------------------------------------
\section{Introduction}\label{sec:introduction}

In April 2023, Bud Light partnered with transgender influencer Dylan Mulvaney for a social media promotion. The backlash was swift and severe: conservative consumers organized boycotts, sales plummeted by 26\%, and the brand lost its position as America's best-selling beer. Yet the company's decision to distance itself from Mulvaney triggered an equally forceful response from progressive consumers and LGBTQ+ advocacy groups, who accused Anheuser-Busch of abandoning its values under pressure. Caught between stakeholder groups with irreconcilable preferences, the company watched its market capitalization decline by billions of dollars \citep{marketing2023bud}.

This episode---and others like it involving Disney, Target, Nike, and Chick-fil-A---illustrates a fundamental challenge facing firms in an era of political polarization: how do organizations manage stakeholders when those stakeholders fundamentally disagree? Stakeholder theory, the dominant paradigm for understanding firm-stakeholder relationships, has traditionally emphasized the possibility of creating value for multiple stakeholders simultaneously \citep{freeman1984strategic, jones1995instrumental}. But this optimistic view assumes that stakeholder interests can be aligned, or at least that trade-offs can be navigated without alienating critical constituencies. When stakeholders hold opposing political views---views tied to identity, values, and social group membership---the possibility of alignment becomes uncertain.

Despite the growing salience of this challenge, we lack systematic evidence about how stakeholder political composition varies across firms. Do some firms serve politically homogeneous customer bases while others navigate divided markets? Are employees and customers of the same firm politically similar, or do they diverge? And when stakeholder groups are misaligned, does it affect firm performance? Answering these questions requires measuring the political composition of firm stakeholders at scale---a task that has been infeasible until recently due to data limitations.

This paper develops novel measures of stakeholder political ideology and uses them to examine ideological alignment across stakeholder groups. We construct measures of \textit{consumer partisan composition} by linking granular foot traffic data to election results: for millions of retail establishments, restaurants, and service locations, we observe which Census Block Groups (CBGs) visitors come from and assign partisan lean based on those neighborhoods' presidential voting patterns. We construct measures of \textit{employee partisan composition} by linking employment records to voter registration data, enabling us to characterize the political orientation of each firm's workforce. Together, these measures provide an unprecedented view of stakeholder ideology across the American economy.

Our empirical approach proceeds in several stages. First, we document descriptive patterns in consumer partisan composition across industries, geographies, and brands. The data reveal substantial variation: while location explains the majority of variance in visitor partisan lean---businesses in Republican areas tend to attract Republican visitors---meaningful differences exist even among businesses in the same neighborhood. Category matters: gun shops and agricultural suppliers attract more conservative visitors than organic grocers and yoga studios, even controlling for location. Brand matters too: Walmart and Target differ in visitor partisan composition beyond what their geographic footprints would predict.

Second, we examine the relationship between employee and consumer political composition. Using a novel entity resolution procedure to link foot traffic data to employment records, we analyze whether the political orientation of a firm's workforce correlates with that of its customers. This analysis speaks directly to stakeholder theory's assumptions about alignment: if employees and customers are systematically similar politically, firms may face less pressure from cross-stakeholder tensions. If they diverge, the challenge of managing conflicting stakeholder preferences becomes more acute.

Third, we examine performance implications. Using SafeGraph Spend data on store-level revenue and exploiting temporal variation in political conflict salience (the Partisan Conflict Index from the Philadelphia Fed), we test whether stakeholder ideological alignment affects firm outcomes. Our identification strategy relies on the intuition that misalignment should matter more when political issues are salient---when the partisan ``temperature'' is high. If stakeholder ideological composition has no effect on firm performance, we would expect the coefficient on misalignment to be zero regardless of political climate. If it matters, we should observe that misalignment hurts performance more during periods of elevated partisan conflict.

Our analysis makes four contributions. First, we develop new measures of stakeholder political composition that can be applied broadly across the economy. Prior research has examined consumer ideology for small samples of firms using surveys \citep{panagopoulos2017all} or inferred preferences from social media \citep{schoenmueller2023polarized}, but no existing measures cover the breadth of commercial activity we observe. Our foot traffic-based approach generates partisan composition estimates for millions of establishments, enabling systematic analysis previously impossible.

Second, we provide the first large-scale empirical documentation of how stakeholder ideology varies across firms. We characterize the distribution of consumer partisan lean across industries, identify which brand categories show the most ideological differentiation, and quantify how much of the variation is explained by location versus business characteristics. These descriptive facts establish a foundation for understanding political sorting in consumer markets.

Third, we examine the relationship between employee and consumer ideology---a question central to stakeholder theory but never empirically addressed at scale. The correlation (or lack thereof) between these stakeholder groups has direct implications for how firms should think about political positioning. Our evidence speaks to whether the ``caught in the middle'' problem is widespread or confined to specific firm types.

Fourth, we provide evidence on the performance consequences of stakeholder ideological alignment. While much research has examined how firms' political \textit{activities}---donations, lobbying, CEO statements---affect outcomes \citep{hillman2004political, hadani2017institutional}, less is known about how stakeholder \textit{composition} influences performance. Our analysis begins to fill this gap, testing whether firms with more aligned stakeholders outperform those navigating divided constituencies.

These contributions extend stakeholder theory in several ways. Most importantly, we introduce ideological heterogeneity as a dimension of stakeholder management. The classic stakeholder framework identifies groups---customers, employees, suppliers, communities---and considers their interests \citep{freeman1984strategic}. But within each group, members may disagree profoundly. A firm's customers are not a monolith; nor are its employees. Understanding \textit{within-stakeholder} diversity, and \textit{across-stakeholder} alignment, enriches the stakeholder perspective and provides practical guidance for managers facing polarized environments.

The paper proceeds as follows. Section~\ref{sec:theory} develops our theoretical framework, engaging with stakeholder theory, the corporate political activity literature, and research on political consumerism. Section~\ref{sec:data} describes our data sources and measure construction. Section~\ref{sec:results} presents empirical findings on descriptive patterns, stakeholder alignment, and performance implications. Section~\ref{sec:discussion} discusses theoretical contributions and limitations. Section~\ref{sec:conclusion} concludes.



% THEORY --------------------------------------------------------------------------
\section{Theory}\label{sec:theory}

We develop a theoretical framework grounded in stakeholder theory that incorporates political ideology as a critical dimension of stakeholder relationships. Our framework engages with three streams of literature: stakeholder theory and its assumptions about alignment, corporate political activity and stakeholder reactions, and research on political consumerism and employee ideology. We then develop predictions about how stakeholder ideological composition varies across firms and how alignment affects firm outcomes.

Our theoretical development speaks to what \citet{bridoux2023new} call the ``new stakeholder theory on organizational purpose,'' which grapples with questions of stakeholder enfranchisement and value distribution. By introducing ideological composition as a dimension of stakeholder relationships, we contribute to understanding how stakeholder preferences---and conflicts among them---shape organizational strategy.

\subsection{Stakeholder Theory and the Alignment Problem}

Stakeholder theory posits that firms should create value for all stakeholders, not just shareholders \citep{freeman1984strategic}. This perspective recognizes that employees, customers, suppliers, communities, and other groups affect and are affected by firm activities, and that successful organizations manage these relationships to generate mutual benefit. The instrumental version of stakeholder theory further argues that attending to stakeholder interests is not merely ethical but also strategically advantageous: firms that treat stakeholders well enjoy better long-term performance \citep{jones1995instrumental}.

A key insight from organizational theory is that managers occupy a central position in reconciling stakeholder claims. \citet{cyert1963behavioral} conceptualized organizations as coalitions of individuals with ``disparate demands, changing foci of attention, and limited ability to attend to all problems simultaneously.'' \citet{pfeffer1978external} extended this view, arguing that organizations are ``other-directed,'' influenced by actors who control critical resources and have managers' attention. \citet{hill1992stakeholder} synthesized these perspectives in their stakeholder-agency model, positioning managers at the center of a ``nexus of contracts'' among stakeholders. In their formulation, managers are unique: they alone enter contractual relationships with all other stakeholders and possess direct control over the firm's decision-making apparatus. Their central position carries responsibility for ``reconciling divergent interests by making strategic decisions and allocating strategic resources in a manner most consistent with the claims of stakeholder groups'' \citep[p.~134]{hill1992stakeholder}.

Central to this framework is the stakeholder salience model developed by \citet{mitchell1997toward}. Salience---the degree to which managers prioritize competing stakeholder claims---depends on three attributes: power (the stakeholder's ability to impose its will), legitimacy (the social appropriateness of the stakeholder's claim), and urgency (the time-sensitivity and criticality of the claim). Stakeholders possessing all three attributes demand immediate managerial attention; those with fewer attributes may be ignored or deprioritized. This model explains why managers attend to some stakeholders more than others, but it assumes that stakeholder preferences are relatively stable and knowable.

Recent work has advanced what \citet{mcgahan2021integrating} calls the ``new stakeholder theory'' (NST), which brings greater precision to understanding stakeholder involvement in organizations. The NST grapples with two canonical questions: which stakeholders are enfranchised in organizations, and how is value created through stakeholder collaboration distributed? \citet{bridoux2022stakeholder} trace how stakeholder theory has evolved from normative arguments about stakeholder consideration to formal analysis of how stakeholder composition affects strategic outcomes. A central insight is that stakeholders bind resources to organizations---employees contribute human capital, customers provide revenue, suppliers offer inputs---and understanding these resource contributions illuminates why stakeholder management is strategically important \citep{barney2018stakeholder}.

An implicit assumption in much stakeholder theory is that stakeholder interests, while diverse, can be reconciled or balanced. \citet{freeman2010stakeholder} argue that value creation need not be zero-sum: clever managers can find solutions that benefit multiple stakeholders simultaneously. This ``both/and'' perspective underpins stakeholder capitalism's normative appeal. Yet it sidesteps a difficult question: what happens when stakeholder interests are not merely different in magnitude but opposite in direction?

Political ideology presents precisely this challenge. When a firm's customers include both liberals and conservatives, satisfying one group's preferences may alienate the other. This is not a problem of incomplete information or misaligned incentives that can be solved through better contracts or communication. It is a fundamental conflict rooted in different values, worldviews, and group identities. The stakeholder framework, developed before polarization reached current levels, does not fully address this scenario.

Our research extends stakeholder theory by introducing ideological heterogeneity as a dimension of stakeholder composition. We examine not only whether firms have customers and employees but what those stakeholders believe politically---and whether different stakeholder groups share similar political orientations. This extension enables us to address the ``caught in the middle'' problem that firms increasingly face.

\subsection{Political Ideology as a Stakeholder Attribute}

Political ideology---an individual's system of beliefs about how society should be organized---shapes economic behavior in multiple ways. A large literature documents that partisan identity influences consumption patterns, employment choices, and reactions to firm activities. These effects operate through several mechanisms.

\textit{Value expression.} Consumers and employees derive utility from associating with organizations that share their values \citep{bhattacharya2003consumer}. Just as consumers may prefer products from firms with strong environmental commitments, they may prefer firms whose political positioning aligns with their own. The rise of ``political consumerism''---using market choices to express political preferences---reflects this tendency \citep{stolle2005consumers, endres2017resisting}. When consumers boycott firms over political stances or buycott firms they wish to support, they are expressing identity through market behavior.

\textit{In-group favoritism.} Social identity theory suggests that individuals favor members of their own groups and disfavor out-groups \citep{tajfel1979integrative}. When partisanship becomes a social identity---as it increasingly has in the United States \citep{mason2018one, iyengar2019origins}---consumers and employees may prefer organizations perceived as co-partisan. This extends beyond explicit political stances: even implicit cues about a firm's political orientation (its customer base, employee composition, geographic footprint) may trigger in-group preferences.

\textit{Trust and legitimacy.} Ideology shapes perceptions of organizational legitimacy. Conservative individuals may view firms with progressive workforces as less trustworthy; liberals may distrust firms associated with conservative causes. Because legitimacy affects willingness to transact---customers must believe the firm will fulfill its promises; employees must believe the firm will treat them fairly---ideological mismatch can erode the foundation of stakeholder relationships.

These mechanisms suggest that stakeholder political composition is not merely a demographic curiosity but a strategically relevant attribute. Firms serving politically homogeneous stakeholder bases may find it easier to maintain trust and satisfy expectations. Firms serving diverse bases face the challenge of appealing to groups with conflicting preferences.

Which mechanism dominates depends on context. Value expression is most salient when consumption or employment is publicly visible and identity-relevant---buying a Tesla signals environmental values; working at a nonprofit signals prosocial orientation. In-group favoritism intensifies when partisan identity is chronically accessible, as during elections or periods of high polarization. Trust and legitimacy concerns arise when stakeholders must make inferences about unobserved firm behavior---will this employer treat me fairly? Will this company honor its warranty? When political cues serve as heuristics for trustworthiness, even incidental signals of ideology can affect transactions.

\subsection{Sorting Equilibria and Stakeholder Selection}

The distribution of stakeholder ideology across firms is not random; it reflects equilibrium sorting processes. Just as labor markets exhibit assortative matching on skills and preferences, consumer and labor markets may exhibit assortative matching on ideology. Understanding these sorting dynamics is essential for interpreting observed patterns of stakeholder composition.

Consider consumer sorting. If liberals prefer firms perceived as progressive, and conservatives prefer firms perceived as traditional, then even small initial differences in firm positioning can be amplified through self-reinforcing dynamics. Early adopters attract like-minded customers, whose presence reinforces the firm's ideological image, attracting more similar customers while deterring dissimilar ones. The result is ideological clustering that exceeds what firm actions alone would produce.

Employee sorting operates through similar mechanisms but with additional frictions. Workers cannot costlessly switch employers; they face search costs, moving costs, and firm-specific human capital that create switching frictions. These frictions mean that employee ideological composition may be stickier than customer composition, adjusting more slowly to changes in firm positioning or local labor market conditions. Geographic constraints also matter: employees must live near their workplace (or accept remote arrangements), while customers can travel or shop online. This geographic tethering means employee ideology may more closely reflect local population ideology than customer ideology does.

The interaction between consumer and employee sorting creates complex equilibrium dynamics. A firm that attracts conservative customers may find it easier to recruit conservative employees (who prefer serving co-partisans), which further reinforces the firm's conservative image. Conversely, a firm with progressive employees may attract progressive customers who value interactions with like-minded workers. These reinforcing dynamics suggest that stakeholder alignment emerges endogenously through sorting, not merely from deliberate firm strategy.

This sorting perspective has important implications. First, observed stakeholder composition reflects both firm choices and stakeholder choices---it is jointly determined. Second, composition may exhibit path dependence: early positioning decisions have persistent effects through cumulative sorting. Third, composition changes face inertia: shifting stakeholder ideology requires overcoming the gravitational pull of existing stakeholders who attract similar others.

\subsection{Corporate Political Activity and Stakeholder Reactions}

A substantial literature examines corporate political activity (CPA)---the actions firms take to influence government policy---and its consequences \citep{hillman2004political}. CPA includes lobbying, political donations, direct engagement with policymakers, and public advocacy on political issues. Research has documented both benefits (favorable policy outcomes, regulatory access) and costs (stakeholder backlash, reputational damage) of political engagement \citep{hadani2017institutional}.

Recent work has increasingly focused on how stakeholders respond to corporate political involvement. \citet{burbano2021effect} finds that employees respond negatively to CEO political donations that conflict with their own preferences, with effects on job satisfaction and turnover intentions. \citet{hambrick2019political} documents that CEO political contributions affect customer perceptions and firm reputation. \citet{melloni2019corporate} shows that firms face pressure from employees and customers when they engage with political issues, and that responses depend on the political composition of these stakeholders.

One stream of this literature examines ``CEO activism''---public statements by corporate leaders on political and social issues \citep{chatterji2019origins}. While CEO activism can enhance brand loyalty among sympathetic customers and increase employee engagement among like-minded workers, it risks alienating stakeholders on the other side. \citet{korschun2022taking} theorizes that CEO activism represents a risky bet: the firm gains among stakeholders who share the CEO's position but loses among those who disagree.

A deeper question is whether CEO activism can be profitable despite this trade-off. \citet{melloni2023cashing} develop a formal model showing that credible, value-enhancing CEO activism is possible, but only under specific conditions. Their key insight is that controversy is a \textit{feature}, not a bug, of effective activism. Consumers understand that CEOs may engage in ``wokewashing''---making false or misleading statements to pander to valuable demographics. For activism to credibly signal firm type, it must be costly enough to deter imitation by firms that do not genuinely share those values. This cost comes from alienating stakeholders on the opposing side. Paradoxically, activism that generates no backlash signals nothing, because it could be costlessly adopted by any firm. Only activism that sacrifices some stakeholders can credibly appeal to others.

This logic extends to competitive dynamics. \citet{mohliver2023corporate} introduce the concept of ``corporate social counterpositioning,'' in which firms deliberately take opposing stances on polarizing social issues to differentiate from competitors. Unlike conventional wisdom that portrays CSR as universally positive, their theory recognizes that highly salient, polarizing issues create opportunities for horizontal differentiation. When a rival adopts a progressive position on gun control or LGBTQ rights, a firm may strategically counterposition by opposing that stance---not despite, but because of, the controversy it generates. The profitability of counterpositioning depends on issue attributes (salience and polarization) and market structure. In fragmented markets where firms compete for distinct customer segments, counterpositioning can be an effective differentiation strategy. This framework helps explain why some firms embrace politically divisive positions rather than pursuing broad, inoffensive CSR.

This research highlights a critical point: the consequences of corporate political engagement depend on stakeholder composition. A firm with a homogeneously liberal customer base may benefit from progressive CEO activism; the same activism could devastate a firm serving conservatives. Recent work by \citet{mckean2024ideologies} examines this directly, finding that ideological alignment between a firm's upper echelons and its stakeholders influences participation in progressive corporate activism. Similarly, \citet{gupta2025political} theorize how growing political polarization reshapes firms' nonmarket strategies over the policy life cycle, showing that polarization delays social consensus and intensifies both activism and industry resistance. Yet existing research typically treats stakeholder composition as unobserved, inferring it indirectly from reactions or modeling it theoretically. We contribute by directly measuring stakeholder ideology, enabling precise analysis of how composition moderates responses to political activity.

\subsection{The Employee Dimension}

While much research on stakeholder politics focuses on consumers, employees represent an equally important stakeholder group. Employees' political beliefs shape their workplace experiences, organizational commitment, and willingness to engage in discretionary effort.

Value congruence---the alignment between individual and organizational values---is a well-established predictor of employee outcomes \citep{kristof1996person, edwards2009person}. When employees perceive that their employer shares their values, they report higher job satisfaction, stronger organizational identification, and lower turnover intentions. Extending this to political values, we would expect employees whose political orientation aligns with their employer's (or their coworkers') to experience greater fit and commitment.

Recent work has begun to examine political dynamics in the workplace directly. \citet{sisco2022managing} finds that political diversity in workplaces can generate conflict, with effects on team performance and employee well-being. \citet{burbano2021effect} documents that employees care about their employers' political activities and may exit when those activities conflict with their beliefs. The Politics at Work project \citep{mcconnell2022economic} reveals substantial variation in employee partisan composition across firms and shows that this composition predicts corporate political behavior.

An important question is whether employee and customer political composition covary. If they do, firms may face reinforcing pressures from aligned stakeholder groups. If they diverge, firms must navigate cross-stakeholder tensions---progressive employees serving conservative customers, or vice versa. This divergence could create internal conflict, affect customer service, and complicate corporate political positioning. We provide the first large-scale evidence on the correlation between employee and customer ideology.

\subsection{Within-Stakeholder Diversity}

Beyond cross-stakeholder alignment, we consider within-stakeholder diversity. A firm's customers are not a homogeneous group; they may themselves hold diverse political views. Similarly, a firm's employees may span the political spectrum. The degree of within-group heterogeneity has implications for stakeholder management.

\citet{harrison2007whats} distinguish three types of diversity: separation (differences in position along a continuum), variety (differences in kind or category), and disparity (differences in concentration of valued resources). Political ideology represents separation diversity---stakeholders differ in their positions along the liberal-conservative spectrum. This distinction matters because separation diversity, unlike variety, tends to generate conflict rather than complementarity. When group members hold opposing positions on valued issues, interaction becomes fraught; compromise positions satisfy no one.

Consider two firms with the same mean customer partisan lean of 50\% Republican. Firm A's customers are uniformly distributed: half are strong Republicans, half are strong Democrats. Firm B's customers cluster near the center: most are moderates or weak partisans. Though both firms have the same average, their strategic situations differ dramatically. Firm A faces the ``caught in the middle'' problem acutely: any political move will alienate half its customers. Firm B, serving a politically moderate base, may have more latitude.

This insight connects to research on market positioning under heterogeneous preferences \citep{caves1977industrial}. Firms may differentiate to serve distinct segments, pursue broad appeal strategies, or attempt niche positioning. When heterogeneity is political, differentiation becomes more fraught: unlike product preferences, political preferences are tied to identity and group membership. Attempting to serve ``both sides'' may be perceived as inauthentic or unprincipled, pleasing neither group.

We therefore examine not only mean stakeholder ideology but its dispersion. Firms with more politically diverse customer bases face different challenges than those serving homogeneous markets, and this diversity may moderate the relationship between ideological alignment and performance.

\subsection{Temporal Variation: The Role of Political Climate}

The salience of political ideology as a stakeholder attribute varies over time. During periods of elevated political conflict---contested elections, Supreme Court decisions, highly publicized political events---partisan identity becomes more accessible and influential. During quieter periods, other considerations may dominate consumer and employee behavior.

This temporal variation provides an identification opportunity. If stakeholder ideological composition affects firm performance, we would expect this effect to be stronger when political issues are salient. When the ``partisan temperature'' is high, consumers are more likely to factor politics into purchasing decisions, employees are more likely to evaluate employers through a political lens, and misalignment should matter more. When the temperature is low, ideology recedes in importance, and alignment or misalignment should have weaker effects.

We exploit this intuition using the Partisan Conflict Index (PCI), a monthly measure of political disagreement developed by the Federal Reserve Bank of Philadelphia \citep{azzimonti2018partisan}. The PCI captures the intensity of partisan conflict based on news coverage of political disagreements. By interacting stakeholder ideological alignment with the PCI, we can test whether alignment matters more during periods of heightened political salience.

This approach addresses endogeneity concerns that plague cross-sectional comparisons. One might worry that firms with more aligned stakeholders differ from misaligned firms in unobservable ways that also affect performance. But if the alignment-performance relationship varies with political climate---stronger when PCI is high, weaker when PCI is low---this suggests the relationship is driven by political mechanisms rather than confounds. The interaction provides quasi-experimental leverage on a fundamentally correlational question.

\subsection{Scope Condition: The Information Environment}

Our theoretical mechanisms require that stakeholders have some awareness of firm ideology or the ideology of fellow stakeholders. This information environment represents an important scope condition. When firm ideology is opaque---unknown to customers, invisible to potential employees---the mechanisms we describe cannot operate. Stakeholders cannot sort on ideology they do not perceive; they cannot reward alignment they do not recognize.

Several factors affect ideological visibility. First, firm size and prominence matter: large public companies receive more media scrutiny, making their political activities and workforce composition more salient. Second, industry context matters: firms in politically charged sectors (energy, firearms, media) face greater attention to their political positioning than firms in neutral sectors. Third, CEO visibility affects transparency: activist CEOs who speak publicly about political issues make firm ideology more observable than CEOs who remain silent.

The rise of social media and online reviews may be increasing ideological transparency across the board. Consumers can now easily learn about firms' political donations, employee reviews can reveal workplace political climate, and viral moments can rapidly publicize a firm's perceived ideology. This increased transparency may be strengthening the mechanisms we describe, making stakeholder ideology more consequential than it was in earlier, less informationally connected eras.

We do not directly test how the information environment moderates our results, but we acknowledge this as a boundary condition. Our findings likely understate the potential effects of ideology in settings with high transparency and overstate them in settings where firm ideology remains hidden from stakeholders.

\subsection{Summary of Theoretical Predictions}

Our theoretical framework generates several predictions, which we organize by the type of analysis required to test them.

First, we expect substantial variation in stakeholder ideological composition across firms. While geography will explain much of this variation---businesses in Republican areas attract Republican visitors---we predict that industry, brand, and business type will also matter. Certain categories (e.g., organic grocers, gun shops) should attract politically distinct customer bases even within the same geographic area.

Second, we expect employee and consumer political composition to be positively correlated but imperfectly aligned. Firms in conservative areas will tend to have both conservative employees and conservative customers, generating positive correlation. But the correlation should be imperfect because employees and customers are selected through different processes: employees choose employers based on job availability, compensation, and career opportunities; customers choose businesses based on product offerings, convenience, and price. These distinct selection processes should introduce divergence.

Third, we expect stakeholder ideological alignment to be associated with firm performance, particularly during periods of elevated political conflict. When employees and customers share political orientations, the firm faces less tension in its stakeholder management and can more easily maintain trust across groups. When they diverge, the firm must navigate conflicting expectations, risking backlash from one group or the other. This relationship should intensify when political issues are salient, as captured by high PCI.

Fourth, we expect within-stakeholder diversity to moderate the alignment-performance relationship, with higher diversity \textit{attenuating} the benefits of alignment. The mechanism is aggregation: when a firm's customers span the political spectrum, positive reactions from aligned customers are offset by negative reactions from misaligned customers, dampening the net effect. Conversely, firms with ideologically homogeneous customer bases face more concentrated reactions---strongly positive when aligned, strongly negative when misaligned. We therefore predict that the alignment-performance relationship is steeper (more positive slope) for firms with low customer ideological diversity than for firms with high diversity.

These predictions guide our empirical analysis. We turn next to describing the data and methods used to test them.



% DATA AND METHODS --------------------------------------------------------------------------
\section{Data and Methods}\label{sec:data}

We combine four primary data sources to construct measures of stakeholder political composition and firm performance: (1) Advan foot traffic data to measure visitor flows, (2) CBG-level presidential election results to characterize visitor political composition, (3) Politics at Work employment data to measure employee political composition, and (4) SafeGraph Spend data to measure store-level performance. We describe each data source and our measure construction procedures below. Detailed methodology, including data processing pipelines and robustness checks, appears in the appendices.

\subsection{Measuring Consumer Political Composition}

\subsubsection{Foot Traffic Data: Advan Monthly Patterns}

We use Advan (formerly SafeGraph) Monthly Patterns data covering January 2019 through July 2024 (67 months) for all 50 U.S. states plus the District of Columbia. The data comprise approximately 596 million POI-month observations across 9.2 million unique points of interest including retail establishments, restaurants, service providers, and other commercial locations.

For each POI-month observation, the data include a unique identifier (PLACEKEY), business attributes (brand, category, NAICS code), location information (address, Census Block Group), and critically, a JSON-encoded field containing the distribution of visitor home CBGs. This field maps each visitor home CBG to a visitor count, enabling us to characterize where a location's visitors reside.

Advan applies differential privacy protections: CBGs with fewer than 4 visitors are suppressed, and visitor counts are subject to noise injection. These protections may introduce measurement error but do not systematically bias partisan lean estimates. In practice, suppressed CBGs account for a small fraction of total visitors.

\subsubsection{Election Data: CBG-Level Presidential Vote Estimates}

We use Census Block Group-level presidential election results derived from the ``Main Method'' approach with RLCR (Registered Voter List with Candidate Records) methodology. These data provide estimated vote counts at the CBG level for both the 2016 and 2020 presidential elections, covering all 283,900 CBGs in the contiguous United States.

The RLCR method combines precinct-level official election returns with geocoded voter file data to produce block group-level vote share estimates. While these estimates are subject to modeling uncertainty, they provide substantially more granular geographic resolution than precinct-level data alone, which is essential for matching to foot traffic visitor distributions.

We compute Republican two-party vote share for each CBG:

\begin{equation}
\text{two\_party\_rep\_share}_{c,t} = \frac{\text{Republican votes}_{c,t}}{\text{Republican votes}_{c,t} + \text{Democratic votes}_{c,t}}
\end{equation}

where $c$ indexes CBGs and $t \in \{2016, 2020\}$ indexes election years. Using two-party vote share excludes third-party votes and creates a bounded measure in $[0, 1]$.

\subsubsection{Constructing Visitor Partisan Lean}

For each POI-month observation, we compute visitor-weighted average Republican vote share:

\begin{equation}
\text{visitor\_rep\_lean}_{i,m,t} = \frac{\sum_{c \in C_{i,m}} \left( \text{rep\_share}_{c,t} \times \text{visitors}_{c,i,m} \right)}{\sum_{c \in C_{i,m}} \text{visitors}_{c,i,m}}
\end{equation}

where $i$ indexes POIs, $m$ indexes months, $t$ indexes election years, and $C_{i,m}$ is the set of visitor home CBGs for POI $i$ in month $m$ that match the election data lookup.

This measure captures the political composition of a location's visitors based on where those visitors live. A store whose visitors primarily come from Republican-leaning neighborhoods will have high visitor partisan lean; one drawing from Democratic-leaning areas will have low lean. The measure reflects realized visitor composition, not underlying preferences, though under reasonable assumptions about residential sorting the two are strongly related.

We compute partisan lean using both 2016 and 2020 election data for all observations, enabling robustness checks across election years. Our primary analyses use 2020 results as the more recent benchmark.

\textit{Full methodology details, including the data processing pipeline and validation procedures, appear in Appendix~\ref{sec:appendix_methodology}.}

\subsection{Measuring Employee Political Composition}

\subsubsection{Employment Data: Politics at Work}

The Politics at Work (PAW) dataset provides employer-level partisan composition measures derived from matching voter registration records to employment histories. The underlying microdata contain approximately 45 million individual employment records with associated voter registration-based partisan identification.

For each employer-year, PAW provides the distribution of employee partisan composition: the share registered as Democrats, Republicans, or unaffiliated/other. We construct employer-level Republican share as:

\begin{equation}
\text{employee\_rep\_share}_{j,y} = \frac{\text{Republican employees}_{j,y}}{\text{Republican employees}_{j,y} + \text{Democratic employees}_{j,y}}
\end{equation}

where $j$ indexes employers and $y$ indexes years. This two-party measure parallels our consumer partisan lean construction.

The PAW data cover approximately 534,000 unique employers from 2012--2024, with extensive coverage of large employers and good representation across industries. Coverage is stronger for firms with substantial formal employment (enabling voter file matching) than for small businesses or those with high workforce turnover.

\subsubsection{Entity Resolution: Linking Advan to PAW}

Linking Advan POI data to PAW employment records requires entity resolution---determining when a business name in one dataset corresponds to the same legal entity in the other. This is challenging because the datasets share no common identifiers: Advan uses PLACEKEYs and brand identifiers; PAW uses company identifiers derived from employer names in voter files.

We develop a tiered matching strategy:

\begin{enumerate}
    \item \textbf{Tier 1: Identifier Matching.} For branded POIs where Advan provides stock tickers, we match directly to PAW records with the same ticker. This approach is precise but covers only publicly traded companies ($\sim$500--800 brands).

    \item \textbf{Tier 2: National Brand Fuzzy Matching.} For remaining branded POIs, we use embedding-based fuzzy matching between Advan brand names and PAW employer names. We generate embeddings using a sentence transformer model and identify high-similarity matches, then manually validate matches above a similarity threshold.

    \item \textbf{Tier 3: Local Business Matching.} For unbranded (``singleton'') POIs, we implement geographically-blocked fuzzy matching, comparing POI names to PAW employers only within the same metropolitan area or county. This approach is more error-prone and is not used in our primary analyses.
\end{enumerate}

Our final matched sample includes 3,872 validated brand-employer linkages covering approximately 1.48 million branded POIs. For these POIs, we can analyze both consumer and employee partisan composition.

\textit{Full entity resolution methodology appears in Appendix~\ref{sec:appendix_entity_resolution}.}

\subsection{Brand-Level Aggregation}

For analyses requiring brand-level measures, we aggregate POI-level visitor partisan lean to the brand level using visit-weighted averages:

\begin{equation}
\text{brand\_rep\_lean}_{b,m,t} = \frac{\sum_{i \in B_b} \left( \text{visitor\_rep\_lean}_{i,m,t} \times \text{visits}_{i,m} \right)}{\sum_{i \in B_b} \text{visits}_{i,m}}
\end{equation}

where $b$ indexes brands, $B_b$ is the set of POIs belonging to brand $b$, and $\text{visits}_{i,m}$ is the normalized visit count (using the \texttt{normalized\_visits\_by\_state\_scaling} field to correct for device sampling variation).

This weighting ensures that high-traffic locations contribute proportionally more to brand-level averages, reflecting their greater importance to brand-level consumer composition.

\subsection{Performance Data: SafeGraph Spend}

To examine performance implications, we use SafeGraph Spend data providing POI-month level transaction information. SafeGraph Spend aggregates anonymized credit and debit card transactions to estimate spending volume at specific locations.

The data include:

\begin{itemize}
    \item \textbf{Spend Amount}: Estimated total spending at the POI in the month
    \item \textbf{Transaction Count}: Number of distinct transactions
    \item \textbf{Customer Count}: Number of unique customers transacting
\end{itemize}

Coverage spans 2019--2021, with strongest coverage for retail and restaurant categories. The data enable us to examine how stakeholder composition relates to store-level economic outcomes.

Limitations include: (1) coverage is incomplete, as not all transactions are captured in the card panel; (2) the data end in 2021, limiting analysis of more recent periods; (3) spend data may be noisy for locations with low transaction volume.

\subsection{Time-Varying Moderator: Partisan Conflict Index}

We use the Partisan Conflict Index (PCI) developed by \citet{azzimonti2018partisan} as a measure of time-varying political salience. The PCI tracks the degree of political disagreement in the United States based on newspaper coverage of political conflict.

The index is constructed by searching major newspaper archives for articles containing terms related to partisan conflict, policy disagreements, or political gridlock. Higher values indicate periods of elevated partisan tension; lower values indicate relative political calm. The index has been validated as a predictor of economic policy uncertainty and investment behavior.

The PCI provides monthly observations from January 1981 through June 2025. We match each POI-month observation to the contemporaneous PCI value, enabling analysis of whether stakeholder alignment effects vary with political climate.

\subsection{Sample Construction and Summary Statistics}

Table~\ref{tab:summary_stats} presents summary statistics for our analysis samples.

\begin{table}[htbp]
\centering
\caption{Summary Statistics}
\label{tab:summary_stats}
\begin{threeparttable}
\begin{tabular}{lrrrr}
\toprule
& Mean & SD & Min & Max \\
\midrule
\multicolumn{5}{l}{\textit{Panel A: POI-Month Level (N = XX million)}} \\
Visitor Republican Lean (2020) & XX.XX & XX.XX & 0.00 & 1.00 \\
Visitor Republican Lean (2016) & XX.XX & XX.XX & 0.00 & 1.00 \\
Monthly Normalized Visits & XX.XX & XX.XX & XX & XX \\
CBG Match Rate (\%) & XX.XX & XX.XX & XX.XX & 100.00 \\
\\
\multicolumn{5}{l}{\textit{Panel B: Brand-Month Level (N = XX thousand)}} \\
Brand Republican Lean (2020) & XX.XX & XX.XX & XX.XX & XX.XX \\
Number of Locations & XX.XX & XX.XX & 1 & XX \\
Total Monthly Visits & XX.XX & XX.XX & XX & XX \\
\\
\multicolumn{5}{l}{\textit{Panel C: Matched Brand-Year Level (N = XX thousand)}} \\
Consumer Republican Lean & XX.XX & XX.XX & XX.XX & XX.XX \\
Employee Republican Lean & XX.XX & XX.XX & XX.XX & XX.XX \\
Employee-Consumer Alignment & XX.XX & XX.XX & XX.XX & XX.XX \\
\\
\multicolumn{5}{l}{\textit{Panel D: Performance Sample (N = XX million)}} \\
Monthly Spend (\$) & XX.XX & XX.XX & XX & XX \\
Transaction Count & XX.XX & XX.XX & XX & XX \\
Partisan Conflict Index & XX.XX & XX.XX & XX.XX & XX.XX \\
\bottomrule
\end{tabular}
\begin{tablenotes}[flushleft]
\small
\item \textit{Notes}: Panel A reports statistics for all POI-month observations with valid visitor CBG data. Panel B aggregates to the brand-month level for branded POIs. Panel C restricts to brands successfully matched to PAW employer records. Panel D restricts to POIs with available SafeGraph Spend data. Visitor Republican Lean is the visitor-weighted average Republican two-party vote share of visitor home CBGs. Employee Republican Lean is the two-party Republican share among registered voters employed at the matched employer.
\end{tablenotes}
\end{threeparttable}
\end{table}

\FloatBarrier



% RESULTS --------------------------------------------------------------------------
\section{Results}\label{sec:results}

We present our empirical findings in four parts. First, we document descriptive patterns in consumer partisan composition across industries, geographies, and brands. Second, we examine brand-level differences in visitor ideology. Third, we analyze the relationship between employee and consumer partisan composition. Fourth, we examine performance implications using SafeGraph Spend data.

\subsection{Descriptive Patterns: Consumer Partisan Composition}

\subsubsection{Geographic Variation}

Consumer partisan lean varies substantially across geography. Figure~\ref{fig:choropleth_map} presents a county-level choropleth map of average visitor Republican lean across all POIs in our sample.

\placeholderfigure{Figure 1: Geographic Distribution of Consumer Partisan Lean}
{County-level choropleth map showing average visitor Republican lean (2020 election). Darker red indicates higher Republican lean; darker blue indicates Democratic lean. Panel A: All POIs. Panel B: Retail only. Panel C: Restaurants only. Expected pattern: Strong geographic sorting consistent with residential partisan clustering, with coastal urban areas showing Democratic lean and interior/rural areas showing Republican lean.}
{Geographic Distribution of Consumer Partisan Lean. Each county is shaded by the average visitor Republican two-party vote share across all POIs in that county. Values range from 0 (entirely Democratic) to 1 (entirely Republican). Data: Advan Monthly Patterns 2019--2024 linked to 2020 presidential election results at the CBG level.}
\label{fig:choropleth_map}

Table~\ref{tab:partisan_by_industry} presents summary statistics on visitor partisan lean by major industry category (NAICS 2-digit).

\begin{table}[htbp]
\centering
\caption{Consumer Partisan Lean by Industry}
\label{tab:partisan_by_industry}
\begin{threeparttable}
\begin{tabular}{lrrrrr}
\toprule
Industry (NAICS 2-digit) & N (POI-months) & Mean & SD & P10 & P90 \\
\midrule
Retail Trade (44-45) & XX,XXX,XXX & XX.XX & XX.XX & XX.XX & XX.XX \\
Accommodation and Food (72) & XX,XXX,XXX & XX.XX & XX.XX & XX.XX & XX.XX \\
Health Care (62) & XX,XXX,XXX & XX.XX & XX.XX & XX.XX & XX.XX \\
Professional Services (54) & XX,XXX,XXX & XX.XX & XX.XX & XX.XX & XX.XX \\
Finance and Insurance (52) & XX,XXX,XXX & XX.XX & XX.XX & XX.XX & XX.XX \\
Arts and Entertainment (71) & XX,XXX,XXX & XX.XX & XX.XX & XX.XX & XX.XX \\
Other Services (81) & XX,XXX,XXX & XX.XX & XX.XX & XX.XX & XX.XX \\
Educational Services (61) & XX,XXX,XXX & XX.XX & XX.XX & XX.XX & XX.XX \\
Real Estate (53) & XX,XXX,XXX & XX.XX & XX.XX & XX.XX & XX.XX \\
Information (51) & XX,XXX,XXX & XX.XX & XX.XX & XX.XX & XX.XX \\
\midrule
\textit{All Industries} & XXX,XXX,XXX & XX.XX & XX.XX & XX.XX & XX.XX \\
\bottomrule
\end{tabular}
\begin{tablenotes}[flushleft]
\small
\item \textit{Notes}: Table reports summary statistics for visitor Republican two-party vote share by NAICS 2-digit industry. N is the count of POI-month observations. Mean is the unweighted average visitor partisan lean. SD is standard deviation. P10 and P90 are the 10th and 90th percentiles. Sample: All POI-months with valid partisan lean estimates, January 2019--July 2024.
\end{tablenotes}
\end{threeparttable}
\end{table}

\subsubsection{Variance Decomposition}

To understand the relative importance of geography versus business characteristics in explaining consumer partisan composition, we decompose variance in visitor partisan lean. We estimate:

\begin{equation}
\text{visitor\_rep\_lean}_{i,m} = \alpha + \delta_{\text{CBG}} + \gamma_{\text{NAICS4}} + \beta_{\text{brand}} + \epsilon_{i,m}
\end{equation}

Table~\ref{tab:variance_decomposition} presents the incremental $R^2$ from adding each fixed effect.

\begin{table}[htbp]
\centering
\caption{Variance Decomposition of Consumer Partisan Lean}
\label{tab:variance_decomposition}
\begin{threeparttable}
\begin{tabular}{lcc}
\toprule
Specification & $R^2$ & Incremental $R^2$ \\
\midrule
Location (CBG) FE only & XX.XX & --- \\
+ Industry (NAICS 4-digit) FE & XX.XX & XX.XX \\
+ Brand FE & XX.XX & XX.XX \\
\midrule
\textit{Expected pattern:} & & \\
Location explains $\sim$85\% & & \\
Industry adds $\sim$5--8\% & & \\
Brand adds $\sim$3--5\% & & \\
\bottomrule
\end{tabular}
\begin{tablenotes}[flushleft]
\small
\item \textit{Notes}: Table reports $R^2$ from regressions of visitor Republican lean on progressively richer sets of fixed effects. Sample: Branded POI-months only. The incremental $R^2$ shows the additional variance explained by each layer of fixed effects.
\end{tablenotes}
\end{threeparttable}
\end{table}

The variance decomposition reveals that while location is the primary determinant of consumer partisan composition, meaningful variation exists at the industry and brand level. Even within the same neighborhood, different types of businesses attract politically distinct customer bases.

\subsection{Brand-Level Patterns}

\subsubsection{Distribution of Brand Partisan Lean}

Figure~\ref{fig:brand_distribution} presents the distribution of brand-level partisan lean across our sample of branded POIs.

\placeholderfigure{Figure 2: Distribution of Brand Partisan Lean}
{Kernel density plot of brand-level Republican lean (visit-weighted average across all locations). X-axis: Republican two-party vote share (0 to 1). Y-axis: Density. Vertical dashed line at 0.5 (neutral). Expected pattern: Roughly normal distribution centered slightly above 0.5, reflecting national Republican advantage in 2020. Modest right skew due to geography of branded retail.}
{Distribution of Brand-Level Consumer Partisan Lean. Each observation is a unique brand's visit-weighted average visitor Republican lean. N = XX,XXX brands with 10+ locations. The dashed vertical line indicates neutral (0.5). Data: Advan Monthly Patterns 2019--2024 aggregated to brand level using normalized visit weights.}
\label{fig:brand_distribution}

\subsubsection{Extreme Brands}

Table~\ref{tab:extreme_brands} presents the 20 brands with the highest and lowest visitor partisan lean, restricted to brands with substantial national presence.

\begin{table}[htbp]
\centering
\caption{Brands by Consumer Partisan Lean (Extremes)}
\label{tab:extreme_brands}
\begin{threeparttable}
\small
\begin{tabular}{llccc}
\toprule
& Brand & Locations & Visits (M) & Rep Lean \\
\midrule
\multicolumn{5}{l}{\textit{Panel A: Most Republican Brands}} \\
1 & [Brand Name] & X,XXX & XX.X & 0.XX \\
2 & [Brand Name] & X,XXX & XX.X & 0.XX \\
3 & [Brand Name] & X,XXX & XX.X & 0.XX \\
4 & [Brand Name] & X,XXX & XX.X & 0.XX \\
5 & [Brand Name] & X,XXX & XX.X & 0.XX \\
... & ... & ... & ... & ... \\
\midrule
\multicolumn{5}{l}{\textit{Panel B: Most Democratic Brands}} \\
1 & [Brand Name] & X,XXX & XX.X & 0.XX \\
2 & [Brand Name] & X,XXX & XX.X & 0.XX \\
3 & [Brand Name] & X,XXX & XX.X & 0.XX \\
4 & [Brand Name] & X,XXX & XX.X & 0.XX \\
5 & [Brand Name] & X,XXX & XX.X & 0.XX \\
... & ... & ... & ... & ... \\
\bottomrule
\end{tabular}
\begin{tablenotes}[flushleft]
\small
\item \textit{Notes}: Table reports the 10 most Republican and 10 most Democratic brands by visitor partisan lean. Sample restricted to brands with 100+ locations and 1M+ annual visits. Rep Lean is the visit-weighted average Republican two-party vote share across all brand locations. Expected extremes: Gun shops, agricultural suppliers, rural retailers on Republican end; organic grocers, yoga studios, specialty coffee on Democratic end.
\end{tablenotes}
\end{threeparttable}
\end{table}

\subsubsection{Within-Category Comparisons}

Beyond overall rankings, instructive comparisons exist within product categories. Table~\ref{tab:category_comparisons} presents partisan lean for competing brands in selected categories.

\begin{table}[htbp]
\centering
\caption{Consumer Partisan Lean: Within-Category Brand Comparisons}
\label{tab:category_comparisons}
\begin{threeparttable}
\begin{tabular}{llcc}
\toprule
Category & Brand & Rep Lean & Difference \\
\midrule
\multicolumn{4}{l}{\textit{Grocery}} \\
& Walmart Supercenter & 0.XX & --- \\
& Kroger & 0.XX & +0.XX \\
& Target & 0.XX & $-$0.XX \\
& Whole Foods & 0.XX & $-$0.XX \\
\midrule
\multicolumn{4}{l}{\textit{Fast Food}} \\
& McDonald's & 0.XX & --- \\
& Chick-fil-A & 0.XX & +0.XX \\
& Chipotle & 0.XX & $-$0.XX \\
\midrule
\multicolumn{4}{l}{\textit{Coffee}} \\
& Starbucks & 0.XX & --- \\
& Dunkin' & 0.XX & +0.XX \\
\midrule
\multicolumn{4}{l}{\textit{Home Improvement}} \\
& Home Depot & 0.XX & --- \\
& Lowe's & 0.XX & +0.XX \\
\bottomrule
\end{tabular}
\begin{tablenotes}[flushleft]
\small
\item \textit{Notes}: Table compares visitor Republican lean across competing brands within selected retail categories. Difference is relative to the reference brand in each category (first listed). All brands restricted to locations with comparable geographic coverage.
\end{tablenotes}
\end{threeparttable}
\end{table}

\subsection{Employee-Consumer Alignment}

We now examine the central question of whether employee and consumer partisan composition are correlated. For the 3,872 brands successfully matched to PAW employer records, we analyze the relationship between the partisan orientation of each brand's visitors and employees.

\subsubsection{Overall Correlation}

Figure~\ref{fig:alignment_scatter} presents a scatter plot of brand-level employee versus consumer partisan lean.

\placeholderfigure{Figure 3: Employee vs. Consumer Partisan Lean (Brand Level)}
{Scatter plot with consumer Republican lean on X-axis and employee Republican lean on Y-axis. Each point is a matched brand. 45-degree line indicates perfect alignment. Expected pattern: Positive correlation ($r \approx$ 0.3--0.5) with substantial scatter. Some brands clustered near 45-degree line (aligned); others showing clear divergence. Notable outliers labeled (e.g., brands where employees are more liberal than customers or vice versa).}
{Employee vs. Consumer Partisan Lean by Brand. Each observation is a brand with matched PAW employer data. X-axis: Consumer Republican lean (visit-weighted average of visitor home CBG 2020 Republican vote share). Y-axis: Employee Republican lean (two-party Republican share from voter registration). N = XX,XXX brand-years. Correlation: $r$ = XX.XX. The 45-degree line indicates perfect employee-consumer alignment.}
\label{fig:alignment_scatter}

Table~\ref{tab:alignment_correlation} reports correlation coefficients overall and within subsamples.

\begin{table}[htbp]
\centering
\caption{Correlation Between Employee and Consumer Partisan Composition}
\label{tab:alignment_correlation}
\begin{threeparttable}
\begin{tabular}{lccc}
\toprule
Sample & N (brand-years) & Pearson $r$ & Spearman $\rho$ \\
\midrule
Full sample & XX,XXX & XX.XX & XX.XX \\
\midrule
\textit{By Industry} & & & \\
Retail Trade & XX,XXX & XX.XX & XX.XX \\
Food Services & XX,XXX & XX.XX & XX.XX \\
Health Care & XX,XXX & XX.XX & XX.XX \\
Professional Services & XX,XXX & XX.XX & XX.XX \\
\midrule
\textit{By Company Size} & & & \\
1,000+ employees & XX,XXX & XX.XX & XX.XX \\
100--999 employees & XX,XXX & XX.XX & XX.XX \\
$<$100 employees & XX,XXX & XX.XX & XX.XX \\
\midrule
\textit{Geographic Controls} & & & \\
Within-MSA correlation & XX,XXX & XX.XX & XX.XX \\
MSA FE residualized & XX,XXX & XX.XX & XX.XX \\
\bottomrule
\end{tabular}
\begin{tablenotes}[flushleft]
\small
\item \textit{Notes}: Table reports correlations between brand-level consumer Republican lean and employee Republican lean. Pearson $r$ is the linear correlation coefficient; Spearman $\rho$ is the rank correlation. Within-MSA correlation restricts to brands operating in a single MSA. MSA FE residualized computes partial correlation after removing MSA fixed effects from both measures. Expected pattern: Positive but modest correlations (0.3--0.5) overall; stronger within industry; weaker after geographic controls.
\end{tablenotes}
\end{threeparttable}
\end{table}

\subsubsection{Patterns of Alignment and Divergence}

To characterize where alignment breaks down, we construct an alignment measure:

\begin{equation}
\text{alignment}_{j} = 1 - \left| \text{consumer\_rep\_lean}_j - \text{employee\_rep\_lean}_j \right|
\end{equation}

Values close to 1 indicate strong alignment (similar consumer and employee composition); values close to 0 indicate maximum divergence. Table~\ref{tab:alignment_patterns} characterizes brands by alignment patterns.

\begin{table}[htbp]
\centering
\caption{Patterns of Employee-Consumer Alignment}
\label{tab:alignment_patterns}
\begin{threeparttable}
\begin{tabular}{lcccc}
\toprule
Pattern & N & Consumer Rep & Employee Rep & Example Brands \\
\midrule
Aligned Republican & XX & 0.6X & 0.6X & [Names] \\
Aligned Neutral & XX & 0.5X & 0.5X & [Names] \\
Aligned Democratic & XX & 0.4X & 0.4X & [Names] \\
Divergent (C$>$E) & XX & 0.6X & 0.4X & [Names] \\
Divergent (E$>$C) & XX & 0.4X & 0.6X & [Names] \\
\bottomrule
\end{tabular}
\begin{tablenotes}[flushleft]
\small
\item \textit{Notes}: Brands categorized by alignment pattern. ``Aligned'' brands have $|$Consumer $-$ Employee$|$ $<$ 0.10. ``Divergent (C$>$E)'' indicates customers are more Republican than employees. ``Divergent (E$>$C)'' indicates employees are more Republican than customers.
\end{tablenotes}
\end{threeparttable}
\end{table}

\subsection{Within-Stakeholder Diversity}

Beyond mean partisan lean, we examine heterogeneity within stakeholder groups. A brand's customers may be uniformly partisan or span the political spectrum. We measure within-stakeholder diversity using the standard deviation of visitor partisan lean across locations and an entropy-based diversity index.

\begin{table}[htbp]
\centering
\caption{Within-Stakeholder Political Diversity by Brand}
\label{tab:within_diversity}
\begin{threeparttable}
\begin{tabular}{lccccc}
\toprule
& & \multicolumn{2}{c}{Consumer Diversity} & \multicolumn{2}{c}{Employee Diversity} \\
\cmidrule(lr){3-4} \cmidrule(lr){5-6}
& N & SD & Entropy & SD & Entropy \\
\midrule
Full sample & XX,XXX & XX.XX & XX.XX & XX.XX & XX.XX \\
\midrule
\textit{By Mean Consumer Lean} & & & & & \\
Republican (mean $>$ 0.55) & XX,XXX & XX.XX & XX.XX & XX.XX & XX.XX \\
Neutral (mean 0.45--0.55) & XX,XXX & XX.XX & XX.XX & XX.XX & XX.XX \\
Democratic (mean $<$ 0.45) & XX,XXX & XX.XX & XX.XX & XX.XX & XX.XX \\
\midrule
\textit{Most Diverse Brands} & & & & & \\
[Brand 1] & --- & XX.XX & XX.XX & XX.XX & XX.XX \\
[Brand 2] & --- & XX.XX & XX.XX & XX.XX & XX.XX \\
\midrule
\textit{Least Diverse Brands} & & & & & \\
[Brand 1] & --- & XX.XX & XX.XX & XX.XX & XX.XX \\
[Brand 2] & --- & XX.XX & XX.XX & XX.XX & XX.XX \\
\bottomrule
\end{tabular}
\begin{tablenotes}[flushleft]
\small
\item \textit{Notes}: Consumer Diversity SD is the standard deviation of visitor Republican lean across POIs within each brand. Entropy is calculated as $-\sum p_k \ln(p_k)$ where $p_k$ is the share of visits from CBGs in partisan quintile $k$. Higher entropy indicates more even distribution across partisan categories. Employee diversity measures computed analogously using PAW data.
\end{tablenotes}
\end{threeparttable}
\end{table}

\subsection{Performance Implications}

We now examine whether stakeholder ideological alignment affects firm performance. Using SafeGraph Spend data, we test whether stores with more aligned stakeholder compositions show stronger economic outcomes.

\subsubsection{Baseline Performance Results}

Table~\ref{tab:performance_baseline} presents OLS estimates relating stakeholder alignment to store-level spending.

\begin{table}[htbp]
\centering
\caption{Stakeholder Alignment and Store Performance}
\label{tab:performance_baseline}
\begin{threeparttable}
\begin{tabular}{lcccc}
\toprule
& (1) & (2) & (3) & (4) \\
Dependent Variable: & \multicolumn{4}{c}{Log Monthly Spend} \\
\midrule
Alignment & XX.XX** & XX.XX** & XX.XX* & XX.XX \\
& (XX.XX) & (XX.XX) & (XX.XX) & (XX.XX) \\
Consumer Rep Lean & & XX.XX & XX.XX & XX.XX \\
& & (XX.XX) & (XX.XX) & (XX.XX) \\
Employee Rep Lean & & XX.XX & XX.XX & XX.XX \\
& & (XX.XX) & (XX.XX) & (XX.XX) \\
\midrule
POI FE & No & No & Yes & Yes \\
Month FE & No & No & No & Yes \\
Local Demographics & No & No & No & Yes \\
\midrule
Observations & XX,XXX,XXX & XX,XXX,XXX & XX,XXX,XXX & XX,XXX,XXX \\
$R^2$ & XX.XX & XX.XX & XX.XX & XX.XX \\
\bottomrule
\end{tabular}
\begin{tablenotes}[flushleft]
\small
\item \textit{Notes}: OLS regressions with dependent variable log monthly spend. Alignment = $1 - |$Consumer Rep Lean $-$ Employee Rep Lean$|$. Standard errors clustered by brand in parentheses. Sample: POI-months with matched SafeGraph Spend and PAW data, 2019--2021. * $p<0.10$, ** $p<0.05$, *** $p<0.01$.
\end{tablenotes}
\end{threeparttable}
\end{table}

\subsubsection{Partisan Climate Moderation}

Our identification strategy exploits temporal variation in the salience of partisan conflict. If stakeholder alignment matters for performance, the effect should be stronger when political issues are salient. Table~\ref{tab:pci_interaction} presents estimates from models interacting alignment with the Partisan Conflict Index (PCI).

\begin{table}[htbp]
\centering
\caption{Partisan Climate Moderation: PCI Interaction}
\label{tab:pci_interaction}
\begin{threeparttable}
\begin{tabular}{lcccc}
\toprule
& (1) & (2) & (3) & (4) \\
Dependent Variable: & \multicolumn{4}{c}{Log Monthly Spend} \\
\midrule
Alignment & XX.XX & XX.XX & XX.XX & XX.XX \\
& (XX.XX) & (XX.XX) & (XX.XX) & (XX.XX) \\
Alignment $\times$ PCI & XX.XX** & XX.XX** & XX.XX* & XX.XX* \\
& (XX.XX) & (XX.XX) & (XX.XX) & (XX.XX) \\
PCI (standardized) & XX.XX & XX.XX & --- & --- \\
& (XX.XX) & (XX.XX) & & \\
\midrule
POI FE & No & Yes & Yes & Yes \\
Month FE & No & No & Yes & Yes \\
Brand $\times$ Year FE & No & No & No & Yes \\
\midrule
Observations & XX,XXX,XXX & XX,XXX,XXX & XX,XXX,XXX & XX,XXX,XXX \\
$R^2$ & XX.XX & XX.XX & XX.XX & XX.XX \\
\bottomrule
\end{tabular}
\begin{tablenotes}[flushleft]
\small
\item \textit{Notes}: OLS regressions examining the interaction between stakeholder alignment and Partisan Conflict Index (PCI). PCI is standardized to have mean 0 and SD 1. A positive coefficient on Alignment $\times$ PCI indicates that alignment matters more during periods of high partisan conflict. Standard errors clustered by brand in parentheses. * $p<0.10$, ** $p<0.05$, *** $p<0.01$.
\end{tablenotes}
\end{threeparttable}
\end{table}

The interaction term tests our core hypothesis: if stakeholder misalignment harms performance, this effect should intensify when partisan conflict is salient. A positive coefficient on Alignment $\times$ PCI indicates that alignment is more valuable (or misalignment more costly) during high-PCI periods.

\subsubsection{Robustness Checks}

Table~\ref{tab:robustness} presents robustness checks for the PCI interaction analysis.

\begin{table}[htbp]
\centering
\caption{Robustness Checks: PCI Interaction}
\label{tab:robustness}
\begin{threeparttable}
\small
\begin{tabular}{lcccc}
\toprule
& Baseline & Lagged PCI & 2016 Election & Transactions \\
& (1) & (2) & (3) & (4) \\
\midrule
Alignment $\times$ PCI & XX.XX** & --- & --- & XX.XX** \\
& (XX.XX) & & & (XX.XX) \\
Alignment $\times$ PCI$_{t-1}$ & --- & XX.XX** & --- & --- \\
& & (XX.XX) & & \\
Alignment (2016) $\times$ PCI & --- & --- & XX.XX** & --- \\
& & & (XX.XX) & \\
\midrule
POI FE, Month FE & Yes & Yes & Yes & Yes \\
Observations & XX,XXX,XXX & XX,XXX,XXX & XX,XXX,XXX & XX,XXX,XXX \\
$R^2$ & XX.XX & XX.XX & XX.XX & XX.XX \\
\bottomrule
\end{tabular}
\begin{tablenotes}[flushleft]
\small
\item \textit{Notes}: Robustness checks. Column (1): baseline specification. Column (2): uses 1-month lagged PCI. Column (3): uses 2016 election data for partisan lean. Column (4): dependent variable is log transaction count. Standard errors clustered by brand in parentheses. * $p<0.10$, ** $p<0.05$, *** $p<0.01$.
\end{tablenotes}
\end{threeparttable}
\end{table}

\subsubsection{Heterogeneity}

Table~\ref{tab:heterogeneity} examines heterogeneity in the alignment-performance relationship across brand and market characteristics.

\begin{table}[htbp]
\centering
\caption{Heterogeneity in Alignment Effects}
\label{tab:heterogeneity}
\begin{threeparttable}
\begin{tabular}{lccc}
\toprule
& Alignment Effect & Alignment $\times$ PCI & N \\
\midrule
\textit{By Industry} & & & \\
Retail Trade & XX.XX & XX.XX & XX,XXX,XXX \\
Food Services & XX.XX & XX.XX & XX,XXX,XXX \\
Other Services & XX.XX & XX.XX & XX,XXX,XXX \\
\midrule
\textit{By Consumer Diversity} & & & \\
High diversity (top quartile) & XX.XX & XX.XX & XX,XXX,XXX \\
Low diversity (bottom quartile) & XX.XX & XX.XX & XX,XXX,XXX \\
\midrule
\textit{By Market Competitiveness} & & & \\
High competition & XX.XX & XX.XX & XX,XXX,XXX \\
Low competition & XX.XX & XX.XX & XX,XXX,XXX \\
\bottomrule
\end{tabular}
\begin{tablenotes}[flushleft]
\small
\item \textit{Notes}: Coefficients from separate regressions by subsample. All specifications include POI and month fixed effects. Consumer diversity measured as standard deviation of visitor partisan lean across locations within brand. Market competitiveness measured using Herfindahl index of brand market shares within CBG. Standard errors omitted for space; ** indicates $p<0.05$.
\end{tablenotes}
\end{threeparttable}
\end{table}

\FloatBarrier

\subsection{Summary of Results}

Our analysis yields four main findings:

\begin{enumerate}
    \item \textbf{Substantial variation in consumer partisan composition.} Visitor partisan lean varies across geography, industry, and brand. While location explains the majority of variance ($\sim$85\%), meaningful differences exist among businesses in the same neighborhood.

    \item \textbf{Brand-level partisan sorting.} Some brands attract systematically more Republican or Democratic visitors than their geographic footprints would predict. Within-category comparisons reveal political differentiation among competing brands.

    \item \textbf{Imperfect employee-consumer alignment.} Employee and consumer partisan composition are positively correlated but imperfectly aligned. The correlation is moderate ($r \approx$ XX.XX), indicating that stakeholder groups are neither perfectly aligned nor independent.

    \item \textbf{Performance implications of alignment.} Stakeholder alignment is associated with store performance, with effects concentrated in periods of high partisan conflict. The interaction between alignment and PCI suggests that misalignment matters more when political issues are salient.
\end{enumerate}



% DISCUSSION --------------------------------------------------------------------------
\section{Discussion}\label{sec:discussion}

This paper introduces stakeholder political pluralism as a new dimension of stakeholder management and provides the first large-scale empirical analysis of how political composition varies across firm stakeholder groups. Our findings have implications for stakeholder theory, corporate strategy, and managerial practice.

\subsection{Theoretical Contributions}

Our primary theoretical contribution is extending stakeholder theory to account for ideological heterogeneity. The classic stakeholder framework identifies stakeholder groups---customers, employees, suppliers, communities---and considers their interests \citep{freeman1984strategic}. But within each group, members may disagree profoundly. A firm's customers are not a monolith with unified preferences; nor are its employees. By introducing stakeholder political pluralism, we enrich the stakeholder perspective to address settings where stakeholders themselves are divided.

This extension speaks directly to the ``new stakeholder theory'' (NST) advanced by \citet{mcgahan2021integrating} and \citet{bridoux2022stakeholder}. The NST grapples with questions of stakeholder enfranchisement and value distribution---which stakeholders matter, and how should firms allocate value among them? Our work adds a prior question: within each stakeholder group, what do members actually want? When stakeholder preferences are heterogeneous, the very notion of ``stakeholder interest'' becomes more complex. A firm cannot simply satisfy ``customer preferences'' when customers prefer opposite things.

Our findings also contribute to the stakeholder salience literature \citep{mitchell1997toward}. Salience theory explains why managers attend to some stakeholders more than others, but implicitly assumes stakeholder preferences are coherent and knowable. When stakeholders are politically pluralistic, salience becomes more nuanced. A customer group may have high power, legitimacy, and urgency, yet be internally divided such that no single response can satisfy the group. Understanding within-stakeholder diversity is thus essential for applying salience concepts.

The stakeholder political pluralism concept also bridges stakeholder theory with the corporate political activity (CPA) literature. CPA research has documented that firms face pressure from stakeholders when engaging in political activities \citep{hillman2004political, burbano2021effect}. Our work clarifies why some firms face more intense pressure than others: pluralistic firms must navigate conflicting expectations from divided stakeholder groups. As \citet{gupta2025political} theorize, polarization intensifies both activism and resistance; our empirical evidence shows that pluralism at the firm level is associated with worse performance outcomes during politically salient periods.

\subsection{Empirical Contributions}

Beyond theory, we make three empirical contributions. First, we develop novel measures of stakeholder political composition that can be applied broadly across the economy. Prior research has examined consumer ideology for limited samples using surveys \citep{panagopoulos2017all} or inferred preferences from social media \citep{schoenmueller2023polarized}, but no existing measures cover the breadth of commercial activity we observe. Our foot traffic-based approach generates partisan composition estimates for millions of establishments, enabling systematic analysis previously impossible.

Second, we provide the first large-scale documentation of how stakeholder political composition varies across firms. We characterize the distribution of consumer partisan lean across industries, identify which brand categories show the most ideological differentiation, and quantify how much variation is explained by location versus business characteristics. The variance decomposition reveals that while geography dominates, meaningful brand-level differentiation exists.

Third, we examine the relationship between employee and consumer ideology---a question central to stakeholder theory but never empirically addressed at scale. Our evidence shows that employee and consumer partisan composition are correlated but imperfectly aligned, with correlation coefficients in the range of [XX.XX to XX.XX]. This moderate correlation indicates that stakeholder pluralism is common: many firms serve customers whose political composition differs from that of their workforce.

\subsection{Implications for Corporate Strategy}

Our findings have direct implications for corporate strategy, particularly regarding political positioning and stakeholder management.

\subsubsection{Political Positioning}

Firms must decide how to position themselves on political issues---or whether to engage at all. The emerging literature on CEO activism and corporate political expression has documented both benefits (enhanced brand loyalty among aligned stakeholders) and risks (backlash from misaligned stakeholders) \citep{chatterji2019origins, korschun2022taking}. Our evidence suggests that the optimal positioning strategy depends critically on stakeholder political composition.

Firms with politically homogeneous stakeholder bases face a clearer calculus: political engagement that aligns with stakeholders' views may be beneficial, while contrary positioning is risky. For pluralistic firms, the calculus is more complex. Any political positioning will alienate some stakeholders. Such firms may benefit from political neutrality or from compartmentalizing messages to different audiences---though both strategies carry risks of their own (perceived inauthenticity, coordination failures).

The PCI interaction results suggest that this calculus intensifies during periods of elevated partisan conflict. When political issues are salient, stakeholder political preferences become more relevant to economic behavior. Pluralistic firms should be especially cautious during these periods, as the costs of misalignment are amplified.

\subsubsection{Geographic Expansion}

Our findings also have implications for geographic expansion decisions. When firms expand into new markets, they acquire new stakeholders whose political composition may differ from existing stakeholders. A brand with a politically homogeneous customer base may face pluralism challenges when entering ideologically different regions.

This suggests that political composition analysis could complement traditional market analysis in expansion decisions. Firms may want to consider not only market size and competitive intensity but also the political composition of potential customers relative to existing stakeholders and employees.

\subsubsection{Workforce Management}

On the employee side, our evidence of imperfect employee-consumer alignment raises questions about workforce composition. Hiring decisions affect not only productivity and wages but also the political composition of the workforce, which in turn affects stakeholder pluralism. Firms operating in politically charged industries may face particular pressure to consider these dynamics.

This is a sensitive area, as political affiliation is protected in some jurisdictions and political hiring criteria would raise serious ethical and legal concerns. We do not advocate for political considerations in hiring. Rather, we note that firms should be aware that workforce composition affects pluralism and should develop strategies for managing internal political diversity constructively \citep{sisco2022managing}.

\subsection{Limitations}

Our analysis has several limitations that suggest caution in interpretation and identify opportunities for future research.

\subsubsection{Ecological Inference}

Our consumer partisan lean measure relies on ecological inference: we infer individual visitors' political preferences from the aggregate voting patterns of their home Census Block Groups. This approach assumes that visitors are politically representative of their neighborhoods, which may not hold if politically motivated sorting occurs within CBGs or if certain political types are more likely to visit commercial establishments.

The ecological inference limitation is common to research using geographically-aggregated data \citep{king1997solution}. We partially address it by using fine-grained CBG-level data (average population $\sim$1,500) rather than larger geographic units, and by focusing on relative comparisons across locations rather than absolute political shares. Nonetheless, individual-level data---if available---would strengthen causal claims.

\subsubsection{Visits vs. Purchases}

Our consumer data capture visits, not purchases. A visitor who enters a store but does not buy differs from a loyal customer. If political sorting occurs more strongly in purchase decisions than visit decisions, our measures may understate true political differentiation in customer bases.

The SafeGraph Spend data partially address this concern by providing transaction-level outcomes, but spend data are available only for a subset of our sample and time period. Future research with richer transaction data could examine whether political composition affects purchase probability conditional on visit.

\subsubsection{Entity Resolution Coverage}

Our employee-consumer alignment analysis relies on entity resolution linking Advan brands to PAW employers. This matching succeeds for 3,872 brands covering 1.48 million POIs, but many brands remain unmatched. If matching success is correlated with pluralism patterns, our alignment estimates may be biased.

We examined selection on observables (brand size, industry, geographic footprint) and found modest differences between matched and unmatched brands. However, we cannot rule out selection on unobservables. The entity resolution challenge is inherent to linking disparate data sources without common identifiers.

\subsubsection{Endogeneity}

Our performance analysis documents associations between stakeholder alignment and economic outcomes, but establishing causality is challenging. Firms with aligned stakeholders may differ from pluralistic firms in unobservable ways that also affect performance. The PCI interaction provides some leverage---if alignment effects vary with an external shock to political salience, this suggests the mechanism is political rather than confounded---but we cannot fully rule out alternative explanations.

Future research could pursue sharper identification strategies, such as examining firm responses to exogenous political events that differentially affect stakeholder groups or exploiting natural experiments in stakeholder composition changes.

\subsection{Future Research Directions}

Our work opens several avenues for future research.

\subsubsection{Worker Mobility and Sorting}

Do workers sort into firms based on political alignment with existing employees or customers? Research on worker mobility could examine whether political composition predicts hiring, turnover, and job search behavior. If workers preferentially join politically aligned organizations, this could reinforce pluralism patterns over time.

\subsubsection{Corporate Political Activity}

How does stakeholder pluralism affect corporate political behavior? Pluralistic firms face pressure from multiple directions when considering political donations, lobbying, or public statements. Research could examine whether pluralism predicts CPA patterns and how firms navigate conflicting stakeholder preferences in political domains.

\subsubsection{Event Studies}

Our temporal analysis uses monthly variation in partisan conflict, but sharper identification might come from event studies around specific political moments---elections, Supreme Court decisions, viral political controversies. How do pluralistic firms' outcomes change around such events relative to aligned firms?

\subsubsection{International Comparisons}

Our analysis focuses on the United States, where political polarization is particularly acute. How does stakeholder political pluralism vary across countries with different political systems and polarization levels? Cross-national research could examine whether our findings generalize beyond the American context.



% CONCLUSION --------------------------------------------------------------------------
\section{Conclusion}\label{sec:conclusion}

Firms today operate in an environment of intense political polarization. Their stakeholders---employees, customers, investors, communities---hold increasingly divergent political views, creating challenges that traditional stakeholder theory did not anticipate. This paper introduces stakeholder political pluralism as a framework for understanding and measuring political heterogeneity within and across stakeholder groups.

Using novel data linking mobile phone location records to election results and employment histories to voter registration, we develop comprehensive measures of consumer and employee political composition for millions of U.S. business establishments. Our analysis reveals substantial variation in stakeholder political composition across geography, industry, and brand. While location explains most of this variation, meaningful differences exist among businesses serving the same neighborhoods. Some brands attract systematically more Republican or Democratic visitors than their geographic footprints would predict; employees and customers are correlated but imperfectly aligned.

These descriptive patterns establish a foundation for understanding political sorting in commercial activity. But the implications extend beyond measurement. Our performance analysis suggests that stakeholder political pluralism matters for firm outcomes, particularly during periods of elevated partisan conflict. When employees and customers diverge politically, and when political issues are salient, misalignment is associated with weaker economic performance. The caught-in-the-middle problem is not merely theoretical; it manifests in measurable outcomes.

For managers, our findings underscore the importance of understanding stakeholder political composition before engaging in political positioning. For a firm with aligned stakeholders, political engagement may be relatively straightforward---the risks are lower, the potential rewards clearer. For pluralistic firms, political engagement requires greater caution. Any position will alienate some stakeholders, and the costs of misalignment intensify during politically charged periods.

More broadly, our work speaks to the challenge of stakeholder management in divided societies. Stakeholder theory's optimistic vision---that firms can create value for all stakeholders simultaneously---assumes that stakeholder interests can be reconciled. When stakeholders disagree fundamentally on political questions, this reconciliation becomes more difficult. Understanding where pluralism exists, and how it affects firm outcomes, is a first step toward developing strategies for navigating these tensions.

We close by noting that political polarization shows few signs of abating. If anything, the integration of politics into consumer identity and workplace dynamics appears to be deepening. The stakeholder political pluralism challenge will likely become more acute, not less. Developing frameworks for understanding and managing this challenge is an important task for management research and practice alike.



% REFERENCES --------------------------------------------------------------------------
\clearpage
\begin{singlespacing}
\bibliographystyle{apalike}
\bibliography{library}
\end{singlespacing}


% APPENDIX --------------------------------------------------------------------------
\clearpage
\appendix

\section{Methodological Appendix: Constructing Visitor Partisan Lean}
\label{sec:appendix_methodology}

\subsection{Overview}

This appendix describes the methodology for constructing measures of visitor partisan lean for points of interest (POIs) across the United States. We link anonymized mobile phone location data to Census Block Group (CBG)-level presidential election results to estimate the political composition of visitors to retail establishments, restaurants, and other commercial locations.

The final output is a panel dataset at the POI-month level containing weighted average Republican vote share of visitors, computed using both 2016 and 2020 presidential election results.

\subsection{Data Sources}

\subsubsection{Foot Traffic Data: Advan Monthly Patterns}

We use Advan (formerly SafeGraph) Monthly Patterns data covering January 2019 through July 2025 (79 months) for all 50 U.S. states plus the District of Columbia. The raw data consists of 2,096 compressed CSV files totaling approximately 804 GB.

For each POI-month observation, the dataset includes a unique identifier (\texttt{PLACEKEY}), business attributes (brand, category, NAICS code), location information (city, state, Census Block Group), and critically, a JSON-encoded field \texttt{VISITOR\_HOME\_CBGS} containing a dictionary mapping visitor home CBGs to visitor counts. For example:

\begin{verbatim}
{"060371234001": 45, "060371234002": 23, "060371235001": 12}
\end{verbatim}

This indicates that 45 visitors came from CBG 060371234001, 23 from CBG 060371234002, and 12 from CBG 060371235001.

Advan applies differential privacy protections: CBGs with fewer than 4 visitors are suppressed, and visitor counts are subject to noise injection. These protections may introduce measurement error but do not systematically bias partisan lean estimates.

\subsubsection{Election Data: CBG-Level Presidential Vote Estimates}

We use CBG-level presidential election results from the ``Main Method'' approach with RLCR (Registered Voter List with Candidate Records) methodology. The data provides estimated vote counts at the Census Block Group level for the 2016 and 2020 presidential elections, covering all 283,900 CBGs in the contiguous United States.

The RLCR method estimates block group-level vote shares by combining precinct-level official election returns with voter file data that includes geocoded addresses and modeled partisanship. This produces more granular estimates than precinct-level data alone, though estimates are subject to modeling uncertainty.

\subsubsection{Metropolitan Statistical Area Crosswalk}

We use the NBER CBSA-to-FIPS County Crosswalk (2023 delineations) to map each POI to its Metropolitan Statistical Area based on the county portion of the POI's CBG code.

\subsection{Data Processing Pipeline}

\subsubsection{Step 1: Construct National CBG Partisan Lean Lookup}

We create a national lookup table containing Republican two-party vote share for each Census Block Group:

\begin{equation}
\text{two\_party\_rep\_share}_{c,t} = \frac{\text{Republican votes}_{c,t}}{\text{Republican votes}_{c,t} + \text{Democratic votes}_{c,t}}
\end{equation}

where $c$ indexes CBGs and $t \in \{2016, 2020\}$ indexes election years. For CBGs with zero total votes, we assign a neutral value of 0.5. The final lookup table contains 283,900 CBGs with partisan lean measures for both election years.

\subsubsection{Step 2: Single-Pass Partisan Lean Computation}

We compute visitor partisan lean using an efficient single-pass architecture. Rather than filtering the raw data by state first (which would require reading each of 2,096 files 51 times for a total of 107,000 file reads), we process each source file exactly once, computing partisan lean for all POIs in that file regardless of state. This reduces total file reads from 107,000 to 2,096---a 52-fold improvement in I/O efficiency.

The processing is implemented as a SLURM array job with 2,096 tasks, one per source file. Each task:

\begin{enumerate}
    \item Loads the national CBG partisan lean lookup into memory as Python dictionaries for $O(1)$ lookup performance
    \item Reads one compressed CSV file, selecting only required columns
    \item For each POI-month observation, parses the \texttt{VISITOR\_HOME\_CBGS} JSON and computes weighted partisan lean
    \item Outputs results to a Parquet file
\end{enumerate}

\subsubsection{Partisan Lean Computation}

For each POI-month observation, we compute the visitor-weighted average Republican vote share:

\begin{equation}
\text{rep\_lean}_{i,m,t} = \frac{\sum_{c \in C_i} \left( \text{rep\_share}_{c,t} \times \text{visitors}_{c,i,m} \right)}{\sum_{c \in C_i} \text{visitors}_{c,i,m}}
\end{equation}

where $i$ indexes POIs, $m$ indexes months, $t$ indexes election years, $C_i$ is the set of visitor home CBGs for POI $i$ that match the election data lookup, and $\text{visitors}_{c,i,m}$ is the count of visitors from CBG $c$ to POI $i$ in month $m$.

Importantly, because we use a \textit{national} CBG lookup table, cross-state visitors are handled correctly. A POI in California with visitors from Nevada will correctly match those Nevada CBGs to their partisan lean values.

We track the match rate for each observation:

\begin{equation}
\text{pct\_matched}_{i,m} = \frac{\text{matched visitors}_{i,m}}{\text{total visitors}_{i,m}} \times 100
\end{equation}

In practice, we achieve match rates of approximately 99\%, with unmatched CBGs typically arising from boundary changes between census years or CBGs in Alaska and Hawaii (which are not in the contiguous USA election data).

\subsubsection{Step 3: Combine and Partition Outputs}

After all 2,096 file-level jobs complete, we combine the outputs into a single national dataset partitioned by year-month for efficient downstream analysis. Summary diagnostics are generated including state-level and brand-level aggregations.

\subsection{Key Methodological Decisions}

\subsubsection{Two-Party Vote Share}

We compute Republican vote share as a proportion of the two-party (Republican + Democratic) vote rather than total votes. This approach excludes third-party votes, creates a bounded measure in $[0, 1]$, and is standard in political science literature.

\subsubsection{Visitor-Count Weighting}

Partisan lean is computed as a visitor-count-weighted average rather than a simple average across CBGs. This ensures that CBGs contributing more visitors have proportionally greater influence on the final measure.

\subsubsection{Both Election Years}

We compute partisan lean using both 2016 and 2020 election data for all POI-months, regardless of the observation date. This allows researchers to assess sensitivity to election year choice and provides robustness checks.

\subsubsection{Handling Unmatched CBGs}

Some visitor home CBGs cannot be matched to the election data lookup due to boundary changes or geographic coverage. We exclude unmatched visitors from the weighted average and track the match rate for each observation, enabling researchers to filter observations with low match rates if desired.

\subsection{Output Data}

The primary output is a POI-month panel with the following key variables:

\begin{itemize}
    \item \texttt{placekey}: Unique POI identifier
    \item \texttt{date\_range\_start}: First day of month
    \item \texttt{brand}: Brand name (null for non-chain POIs)
    \item \texttt{region}: Two-letter state code
    \item \texttt{cbsa\_title}: Metropolitan Statistical Area name
    \item \texttt{rep\_lean\_2020}: Weighted Republican vote share (2020 election), range $[0, 1]$
    \item \texttt{rep\_lean\_2016}: Weighted Republican vote share (2016 election), range $[0, 1]$
    \item \texttt{total\_visitors}: Total visitor count from all CBGs
    \item \texttt{matched\_visitors}: Visitor count from CBGs with election data
    \item \texttt{pct\_visitors\_matched}: Match rate percentage
\end{itemize}

\subsection{Computational Implementation}

The pipeline is implemented on the UC Berkeley Savio HPC cluster using SLURM for job scheduling. The single-pass architecture processes 2,096 source files in parallel, with each task completing in approximately 45 seconds. Total wall-clock time for the full pipeline is approximately 1-2 hours with 50 concurrent tasks.

\subsection{Limitations}

\begin{enumerate}
    \item \textbf{Mobile phone sampling:} The Advan data may not be representative of all visitors due to differential smartphone adoption rates.
    \item \textbf{Privacy-preserving noise:} Differential privacy protections may attenuate measured partisan differences.
    \item \textbf{Ecological inference:} We observe aggregate CBG-level voting patterns, not individual preferences.
    \item \textbf{Home location inference:} The ``home'' CBG is inferred from nighttime location patterns and may not reflect current residence for all individuals.
\end{enumerate}


%%%%%%%%%%%%%%%%%%%%%%%%%%%%%%%%%%%%%%%%%%%%%%%%%%%%%%%%%%%%%%%%%%%%%%%%%%%%%%%
% ENTITY RESOLUTION METHODOLOGY - EXPANDED APPENDIX
% Linking Advan POIs to Politics at Work Employers
%%%%%%%%%%%%%%%%%%%%%%%%%%%%%%%%%%%%%%%%%%%%%%%%%%%%%%%%%%%%%%%%%%%%%%%%%%%%%%%

\section{Methodological Appendix: Entity Resolution for Employee Ideology Linkage}
\label{sec:appendix_entity_resolution}

\subsection{Overview}

This appendix describes the methodology for linking points of interest (POIs) from the Advan foot traffic data to employer records in the Politics at Work (PAW) dataset. This linkage enables joint analysis of employee partisan composition alongside visitor partisan composition, which is central to our research questions about stakeholder ideological alignment.

The core challenge is \textit{entity resolution}: determining when a business name in Advan (e.g., ``Starbucks'' or ``Joe's Pizza'') corresponds to the same legal entity as an employer record in PAW. Unlike datasets with common identifiers (GVKEY, EIN, DUNS), Advan and PAW share no standard company identifiers, necessitating name-based matching.

We employ a \textit{tiered matching strategy} that combines exact identifier matching (where available), neural embedding-based fuzzy matching for national brands, and geographically-blocked fuzzy matching for local businesses. This approach balances match coverage against precision by applying increasingly sophisticated (and error-prone) methods only where simpler methods fail.

\subsection{Data Sources}

\subsubsection{Advan POI Data}

The Advan Places of Interest (POI) dataset provides location-level data for commercial establishments. For entity resolution, the relevant fields are:

\begin{table}[htbp]
\centering
\caption{Advan POI Fields for Entity Resolution}
\label{tab:advan_fields}
\begin{tabular}{lll}
\toprule
Field & Type & Description \\
\midrule
\texttt{PLACEKEY} & string & Unique POI identifier \\
\texttt{SAFEGRAPH\_BRAND\_IDS} & string & Brand identifier (chain locations only) \\
\texttt{BRANDS} & string & Brand name (e.g., ``Starbucks'') \\
\texttt{LOCATION\_NAME} & string & Business name (all POIs) \\
\texttt{NAICS\_CODE} & string & 6-digit NAICS industry code \\
\texttt{CITY} & string & City name \\
\texttt{REGION} & string & State abbreviation \\
\texttt{POSTAL\_CODE} & string & ZIP code \\
\texttt{POI\_CBG} & string & 12-digit Census Block Group code \\
\texttt{LATITUDE} & float & Geographic coordinate \\
\texttt{LONGITUDE} & float & Geographic coordinate \\
\bottomrule
\end{tabular}
\end{table}

Note that Advan does \textit{not} provide CBSA (Core-Based Statistical Area) or MSA identifiers directly. We derive MSA assignment from the \texttt{POI\_CBG} field by extracting the county FIPS code (first 5 digits of the 12-digit CBG code) and joining to the NBER CBSA-to-FIPS crosswalk, as detailed in Section~\ref{sec:poi_msa_assignment}.

After filtering to the 50 U.S. states plus the District of Columbia, the POI population comprises approximately 19.8 million locations:

\begin{itemize}
    \item \textbf{Branded POIs}: 2.35 million locations ($\sim$12\%) with \texttt{SAFEGRAPH\_BRAND\_IDS} populated, mapping to approximately 14,800 unique brands
    \item \textbf{Unbranded (``singleton'') POIs}: 17.5 million locations ($\sim$88\%) without brand identifiers, representing independent businesses, franchises without brand coding, or establishments not recognized as chains
\end{itemize}

This distinction is critical for our matching strategy: branded POIs can be matched at the brand level (one match propagates to thousands of locations), while singleton POIs must be matched individually.

\subsubsection{Advan Brand Information}

The Advan Brand Info file provides brand-level metadata:

\begin{table}[htbp]
\centering
\caption{Advan Brand Info Fields}
\begin{tabular}{lll}
\toprule
Field & Type & Description \\
\midrule
\texttt{SAFEGRAPH\_BRAND\_ID} & string & Unique brand identifier \\
\texttt{BRAND\_NAME} & string & Brand name \\
\texttt{STOCK\_SYMBOL} & string & Stock ticker (public companies) \\
\texttt{NAICS\_CODE} & string & Primary NAICS code \\
\texttt{PARENT\_SAFEGRAPH\_BRAND\_IDS} & string & Parent brand identifiers \\
\bottomrule
\end{tabular}
\end{table}

The brand file contains approximately 14,800 unique brands. Of these, approximately 500--800 have stock symbols populated, enabling direct ticker-based matching to public companies.

\subsubsection{Politics at Work (PAW) Employment Data}

The Politics at Work dataset provides employer-level partisan composition measures derived from voter registration records matched to employment histories. The employer-year panel contains 6.26 million employer-year observations across 534,000+ unique employers from 2012--2024.

For entity resolution, the relevant fields are:

\begin{table}[htbp]
\centering
\caption{PAW Employer Fields for Entity Resolution}
\label{tab:paw_fields}
\begin{tabular}{lll}
\toprule
Field & Type & Description \\
\midrule
\texttt{rcid} & string & Revelio company identifier (stable across years) \\
\texttt{company\_name} & string & Employer name \\
\texttt{ticker} & string & Stock ticker (available for $\sim$0.2\% of records) \\
\texttt{gvkey} & string & Compustat identifier (public companies) \\
\texttt{naics\_code} & string & NAICS industry code ($\sim$96\% populated) \\
\texttt{modal\_msa} & string & Company's primary MSA \\
\texttt{ultimate\_parent\_rcid} & string & Ultimate parent company identifier \\
\texttt{employee\_count} & integer & Total employee count \\
\bottomrule
\end{tabular}
\end{table}

PAW partisan composition variables (not used for matching, but linked post-resolution):

\begin{itemize}
    \item \texttt{pct\_dem\_imp}, \texttt{pct\_rep\_imp}: Imputed Democratic/Republican employee percentages
    \item \texttt{two\_party\_margin\_imp}: Democratic minus Republican margin
    \item \texttt{political\_diversity\_imp}: Herfindahl-based diversity measure
    \item \texttt{effective\_parties\_imp}: Inverse Herfindahl index
\end{itemize}

After deduplication on company name, PAW contains approximately 26.3 million unique company name strings. However, the effective matching universe is much smaller when filtered to companies with sufficient employee counts for reliable ideology estimates.

\subsubsection{Geographic Crosswalks}

We use two geographic crosswalks for MSA-based blocking:

\begin{enumerate}
    \item \textbf{NBER CBSA-to-FIPS Crosswalk}: Maps county FIPS codes to Core-Based Statistical Areas (CBSAs), using 2023 Census Bureau delineations. This enables mapping Advan POIs to MSAs via the county portion of their Census Block Group code.

    \item \textbf{PAW MSA Position Files}: Pre-computed files containing all PAW employers with at least one employee in each MSA. These files naturally encode which companies operate in which metropolitan areas, enabling efficient geographic blocking without explicit company-MSA crosswalk construction.
\end{enumerate}

\subsubsection{Data Access and Retrieval Dates}

Table~\ref{tab:data_access} documents the access dates for all data sources used in this analysis. These dates are provided for replication purposes and to establish the temporal snapshot of each dataset.

\begin{table}[htbp]
\centering
\caption{Data Access Dates}
\label{tab:data_access}
\begin{tabular}{llll}
\toprule
Data Source & Access Date & Provider & Notes \\
\midrule
Advan Monthly Patterns & January 12, 2026 & Dewey Data & Coverage: Jan 2019--Jul 2025 \\
Advan Brand Info & January 15, 2026 & Dewey Data & Derived from POI data \\
Politics at Work (PAW) & April 16, 2025 & Harvard Dataverse & Revelio employment records \\
CBG Election Results & December 29, 2025 & Enamorado et al. & 2016 and 2020 presidential \\
NBER CBSA Crosswalk & January 13, 2026 & NBER & 2023 Census delineations \\
\bottomrule
\end{tabular}
\end{table}

All data were accessed under appropriate data use agreements. Advan data were obtained through a commercial license via Dewey Data. Politics at Work data are available through Harvard Dataverse with restricted access for research purposes. CBG-level election results are from the geocoded election returns dataset. The NBER CBSA crosswalk is publicly available.

\subsection{Tiered Matching Strategy}

We employ a five-tier matching strategy, applying increasingly sophisticated methods to progressively smaller subsets of unmatched records. Each tier is designed to maximize precision for its target population while acknowledging that later tiers necessarily trade precision for coverage.

\begin{table}[htbp]
\centering
\caption{Entity Resolution Tiers}
\label{tab:matching_tiers}
\begin{tabular}{llrrr}
\toprule
Tier & Method & Scope & Expected Precision & Expected Coverage \\
\midrule
0a & Exact ticker match & Public brands & 100\% & 500--800 brands \\
0b & GVKey via ticker & Public brands & 100\% & Same as 0a \\
1 & Brand embedding match & All brands & 90--95\% & 80\%+ of brands \\
2 & Singleton MSA-blocked & Unbranded POIs & 70--80\% & 20--40\% of singletons \\
3 & String distance fallback & Remaining & 50--70\% & Marginal \\
\bottomrule
\end{tabular}
\end{table}

\subsubsection{Tier 0a: Exact Ticker Matching}

For Advan brands with stock symbols and PAW companies with tickers, we perform exact string matching:

\begin{equation}
\text{match}(i, j) = \mathbbm{1}\left[\text{Advan.STOCK\_SYMBOL}_i = \text{PAW.ticker}_j\right]
\end{equation}

This produces high-confidence matches for major public companies. Approximately 500--800 Advan brands have stock symbols, representing the largest national chains (Walmart, McDonald's, Starbucks, Target, etc.). These brands account for a disproportionate share of POIs---the top 100 brands by location count cover approximately 15\% of all branded POIs.

\textbf{Match quality}: 100\% precision by construction (exact string match on standardized identifiers).

\textbf{Cost}: Zero (no API calls required).

\subsubsection{Tier 0b: GVKey Extraction via Ticker}

For brands matched in Tier 0a, we extract the Compustat \texttt{gvkey} identifier from PAW records. This enables downstream linkage to financial performance data (Compustat) for public company analysis.

Note that Advan does not provide GVKey directly; the ticker serves as a bridge identifier. PAW records with tickers generally have GVKeys populated, making this a one-step lookup.

\subsubsection{Tier 1: Brand Name Matching via Neural Embeddings}

For unmatched brands (those without stock symbols or without ticker matches), we employ fuzzy name matching using neural text embeddings. This approach handles the substantial variation in company name representations:

\begin{itemize}
    \item Corporate suffixes: ``Starbucks'' vs. ``Starbucks Corporation'' vs. ``Starbucks Inc.''
    \item Abbreviations: ``McDonald's'' vs. ``McDonalds'' vs. ``MCD''
    \item Parent vs. brand: ``Yum! Brands'' vs. ``Taco Bell'' vs. ``KFC''
    \item Punctuation and spacing: ``Dunkin' Donuts'' vs. ``Dunkin Donuts''
\end{itemize}

\paragraph{Embedding Generation}

We generate vector embeddings for each unique company name using OpenAI's \texttt{text-embedding-3-small} model, which produces 1,536-dimensional vectors optimized for semantic similarity tasks. We choose this model over larger alternatives because:

\begin{enumerate}
    \item Company names are short strings (typically 2--5 words) where larger models provide marginal benefit
    \item Cost is significantly lower (\$0.02 per million tokens vs. \$0.13 for \texttt{text-embedding-3-large})
    \item Latency is lower, enabling faster processing of large batches
\end{enumerate}

\paragraph{Similarity Computation}

For each Advan brand $i$ and PAW company $j$, we compute cosine similarity between their embedding vectors:

\begin{equation}
\text{sim}(i, j) = \frac{\mathbf{e}_i \cdot \mathbf{e}_j}{\|\mathbf{e}_i\| \|\mathbf{e}_j\|}
\end{equation}

where $\mathbf{e}_i$ and $\mathbf{e}_j$ are the embedding vectors for Advan brand $i$ and PAW company $j$ respectively.

\paragraph{Candidate Selection and Thresholding}

For each Advan brand, we identify the PAW company with highest cosine similarity and apply acceptance thresholds:

\begin{itemize}
    \item \textbf{Accept}: $\text{sim} > 0.85$ --- Sufficient similarity for match
    \item \textbf{Reject}: $\text{sim} \leq 0.85$ --- Insufficient similarity
\end{itemize}

When multiple PAW companies have similar scores (within 0.05 of the maximum), we use Jaro-Winkler string distance as a secondary signal to break ties. Jaro-Winkler gives additional weight to prefix matches, which is appropriate for company names where the distinctive information often appears first.

\paragraph{NAICS Code Validation}

For candidate matches, we compare NAICS codes at the 4-digit level when available in both datasets. NAICS agreement provides a sanity check: a restaurant brand should not match to a manufacturing company. Candidates with NAICS disagreement at the 2-digit level are excluded from matching.

\paragraph{Computational Cost}

\begin{table}[htbp]
\centering
\caption{Tier 1 Embedding Costs (Estimated)}
\begin{tabular}{lrrr}
\toprule
Component & Records & Tokens (est.) & Cost \\
\midrule
Advan brand names & 14,800 & 45,000 & \$0.001 \\
PAW company names (filtered) & 50,000 & 150,000 & \$0.003 \\
\textbf{Total} & & & \textbf{\$0.004} \\
\bottomrule
\end{tabular}
\end{table}

We filter PAW companies to those with $\geq$50 employees to focus matching efforts on companies with reliable ideology estimates. This reduces the PAW candidate set from 26.3 million to approximately 50,000 companies while retaining the most analytically relevant employers.

\subsubsection{Tier 2: Singleton POI Matching with MSA Blocking}

For the 17.5 million unbranded POIs, we employ fuzzy matching with geographic blocking by Metropolitan Statistical Area (MSA). Geographic blocking is essential because:

\begin{enumerate}
    \item Singleton POIs represent local businesses, not national chains
    \item A ``Joe's Pizza'' in New York is almost certainly different from ``Joe's Pizza'' in Los Angeles
    \item Without blocking, the comparison space would be intractable: $17.5\text{M} \times 26.3\text{M} = 4.6 \times 10^{14}$ pairs
\end{enumerate}

\paragraph{POI-to-MSA Assignment}
\label{sec:poi_msa_assignment}

We assign each Advan POI to an MSA using the \texttt{POI\_CBG} field, which contains the 12-digit Census Block Group code. The first 5 digits encode the county FIPS code:

\begin{equation}
\text{county\_fips} = \text{POI\_CBG}[0:5]
\end{equation}

We join this to the NBER CBSA-to-FIPS crosswalk to obtain the CBSA (MSA/micropolitan area) code:

\begin{verbatim}
POI_CBG: "060371234001"
          ↓
County FIPS: "06037" (Los Angeles County, CA)
          ↓
CBSA Code: "31080" (Los Angeles-Long Beach-Anaheim, CA)
\end{verbatim}

Approximately 15--20\% of POIs are located in counties not part of any CBSA (rural areas). These POIs cannot be MSA-blocked and are handled separately (see Section~\ref{sec:rural_pois}).

\paragraph{PAW Company-MSA Extraction}

The PAW MSA position files (one per MSA) contain all employers with at least one employee in that MSA. This naturally encodes multi-MSA company presence: a regional company operating in three MSAs appears in all three position files.

For each MSA, we extract the unique set of company identifiers and names:

\begin{verbatim}
For MSA X:
  Load {msa_x}_positions.parquet
  Extract unique (rcid, company_name) pairs
  Result: all companies operating in MSA X
\end{verbatim}

\paragraph{Within-MSA Matching}

For each MSA, we perform embedding-based matching between:
\begin{itemize}
    \item \textbf{Query set}: Unbranded POI \texttt{LOCATION\_NAME} values in that MSA
    \item \textbf{Candidate set}: PAW \texttt{company\_name} values from that MSA's position file
\end{itemize}

We apply stricter thresholds than Tier 1, reflecting higher noise in singleton matching:

\begin{itemize}
    \item \textbf{Auto-accept}: $\text{sim} > 0.92$
    \item \textbf{Accept with NAICS match}: $0.88 < \text{sim} \leq 0.92$ and NAICS 4-digit agreement
    \item \textbf{Reject}: $\text{sim} \leq 0.88$ or NAICS disagreement at 2-digit level
\end{itemize}

\paragraph{Parallelization Strategy}

Tier 2 is implemented as a SLURM array job with one task per MSA. The 384 MSAs are processed independently, with outputs written to MSA-specific Parquet files. This enables:

\begin{enumerate}
    \item Embarrassingly parallel execution (no cross-MSA dependencies)
    \item Memory efficiency (each task loads only one MSA's data)
    \item Incremental progress (partial failures don't require full restart)
\end{enumerate}

\paragraph{Computational Cost}

\begin{table}[htbp]
\centering
\caption{Tier 2 Embedding Costs (Estimated, Full Scale)}
\begin{tabular}{lrrr}
\toprule
Component & Records & Tokens (est.) & Cost \\
\midrule
Singleton POI names & 17,500,000 & 52,500,000 & \$1.05 \\
PAW company names (all MSAs) & 2,000,000 & 6,000,000 & \$0.12 \\
\textbf{Total} & & & \textbf{\$1.17} \\
\bottomrule
\end{tabular}
\end{table}

The actual cost is lower because many POI names are duplicates across locations (e.g., ``Hair Salon'' appears thousands of times), and we embed unique strings only.

\subsubsection{Tier 3: String Distance Fallback}

For POIs unmatched after Tiers 0--2, we apply traditional string distance methods as a fallback:

\begin{itemize}
    \item \textbf{Jaro-Winkler distance}: Emphasizes prefix matches, handles transpositions
    \item \textbf{Jaccard similarity on tokens}: Computes overlap of word tokens
    \item \textbf{Levenshtein distance}: Edit distance for character-level matching
\end{itemize}

These methods serve primarily as robustness checks and to capture matches that embeddings miss (e.g., substantial abbreviations like ``Intl'' for ``International'').

\textbf{Expected yield}: Marginal additional matches beyond Tiers 0--2.

\textbf{Cost}: Zero (local computation only).

\subsection{Handling Special Cases}

\subsubsection{Multi-MSA Companies}
\label{sec:multi_msa}

Companies operating across multiple MSAs naturally appear in multiple MSA position files. When processing MSA $X$, we match POIs in $X$ against all companies with employees in $X$---including regional and national companies.

This design ensures that:
\begin{enumerate}
    \item A ``Joe's Pizza'' POI in Columbus matches against ``Joe's Pizza'' in the Columbus position file (if it has employees there)
    \item The same company may be matched in multiple MSAs, producing consistent \texttt{rcid} linkages
    \item No explicit company-MSA footprint computation is required
\end{enumerate}

\subsubsection{Franchise vs. Corporate Matching}
\label{sec:franchises}

Many branded POIs are franchises with independent ownership. In PAW, these may appear as:
\begin{itemize}
    \item The parent brand (``McDonald's Corporation'')
    \item Individual franchise entities (``Golden Arches LLC'')
\end{itemize}

For Tier 1 (brand matching), we match to the parent brand, providing corporate-level employee ideology. For Tier 2 (singleton matching), we may match to individual franchise entities, providing location-specific employee ideology.

We retain both linkages where available, enabling sensitivity analysis: does brand-level or franchise-level employee composition better predict visitor composition?

\subsubsection{Branded POIs Without Brand IDs}

Some POIs that are clearly chain locations lack \texttt{SAFEGRAPH\_BRAND\_IDS} due to data quality issues. These are coded as singletons and matched via Tier 2.

After completing Tier 1, we propagate brand$\rightarrow$rcid mappings to Tier 2: if a singleton POI matches to the same \texttt{rcid} as a known brand, we flag it as a likely uncoded brand location.

\subsubsection{Rural POIs (Non-MSA Locations)}
\label{sec:rural_pois}

Approximately 15--20\% of POIs are located in counties not part of any CBSA. For these POIs:

\begin{enumerate}
    \item \textbf{Primary approach}: Match against statewide company list (all companies with employees in that state)
    \item \textbf{Alternative}: Use micropolitan area definitions where available
    \item \textbf{Fallback}: Exclude from MSA-blocked matching; include only Tier 0--1 matches
\end{enumerate}

Rural POIs are flagged with a data quality indicator. Researchers may choose to restrict analysis to MSA-located POIs for cleaner matching.

\subsubsection{Temporal Name Changes}

Company names change over time due to mergers, acquisitions, and rebranding. PAW records span 2012--2024; Advan data spans 2019--2025. We address temporal misalignment by:

\begin{enumerate}
    \item Matching against the most recent PAW company name
    \item Using \texttt{ultimate\_parent\_rcid} to link subsidiaries that may have changed names
    \item Flagging matches where Advan and PAW names differ substantially (Jaro-Winkler $< 0.8$) as lower confidence
\end{enumerate}

\subsection{Output Schema}

\subsubsection{Brand Crosswalk}

For Tier 0--1 matches (brand-level), we output \texttt{brand\_crosswalk.parquet}:

\begin{table}[htbp]
\centering
\caption{Brand Crosswalk Schema}
\begin{tabular}{lll}
\toprule
Column & Type & Description \\
\midrule
\texttt{safegraph\_brand\_id} & string & Advan brand identifier \\
\texttt{brand\_name} & string & Advan brand name \\
\texttt{stock\_symbol} & string & Stock ticker (if available) \\
\texttt{rcid} & string & PAW company identifier \\
\texttt{company\_name} & string & PAW company name \\
\texttt{gvkey} & string & Compustat identifier (if available) \\
\texttt{ultimate\_parent\_rcid} & string & Parent company identifier \\
\texttt{match\_tier} & string & ``ticker\_exact'' or ``brand\_embedding'' \\
\texttt{cosine\_similarity} & float & Embedding similarity (Tier 1 only) \\
\texttt{naics\_match} & boolean & NAICS codes agree at 4-digit level \\
\bottomrule
\end{tabular}
\end{table}

\subsubsection{Singleton Crosswalk}

For Tier 2 matches (location-level), we output \texttt{singleton\_crosswalk/\{msa\_code\}.parquet}:

\begin{table}[htbp]
\centering
\caption{Singleton Crosswalk Schema}
\begin{tabular}{lll}
\toprule
Column & Type & Description \\
\midrule
\texttt{placekey} & string & Advan POI identifier \\
\texttt{location\_name} & string & Advan location name \\
\texttt{poi\_cbg} & string & Census Block Group \\
\texttt{msa\_code} & string & CBSA code \\
\texttt{msa\_name} & string & CBSA name \\
\texttt{rcid} & string & PAW company identifier \\
\texttt{company\_name} & string & PAW company name \\
\texttt{cosine\_similarity} & float & Embedding similarity score \\
\texttt{naics\_match} & boolean & NAICS codes agree at 4-digit level \\
\texttt{is\_likely\_uncoded\_brand} & boolean & Matches known brand rcid \\
\bottomrule
\end{tabular}
\end{table}

\subsubsection{Master POI-to-Employer Crosswalk}

The final output combines all tiers into \texttt{poi\_employer\_crosswalk.parquet}:

\begin{table}[htbp]
\centering
\caption{Master Crosswalk Schema}
\begin{tabular}{lll}
\toprule
Column & Type & Description \\
\midrule
\texttt{placekey} & string & Advan POI identifier \\
\texttt{safegraph\_brand\_id} & string & Brand ID (branded POIs only) \\
\texttt{rcid} & string & PAW company identifier \\
\texttt{ultimate\_parent\_rcid} & string & Parent company identifier \\
\texttt{gvkey} & string & Compustat identifier (public companies) \\
\texttt{match\_tier} & string & Which tier produced the match \\
\texttt{match\_confidence} & float & Composite confidence score \\
\texttt{match\_source} & string & ``brand'' or ``singleton'' \\
\bottomrule
\end{tabular}
\end{table}

\subsection{Pipeline Implementation}

The following computational resources and costs are \textit{estimates} based on data sizes and API pricing as of January 2026. Actual values may vary based on final data volumes and API pricing changes.

\subsubsection{Computational Resources}

\begin{table}[htbp]
\centering
\caption{SLURM Resource Allocation by Step}
\begin{tabular}{llll}
\toprule
Step & Partition & Rationale & Wall Time (est.) \\
\midrule
Extract unique entities & savio3 & Moderate memory for scans & 30 min \\
Tier 0: Ticker match & savio2 & Trivial join & 5 min \\
Tier 1: Brand embeddings & savio2 & API calls, not compute-intensive & 1 hour \\
Tier 2: Singleton (per MSA) & savio3 array & Parallel by MSA, 384 tasks & 2--4 hours \\
Combine crosswalks & savio3\_bigmem & Large join operations & 1 hour \\
\bottomrule
\end{tabular}
\end{table}

\subsubsection{API Cost Summary}

\begin{table}[htbp]
\centering
\caption{Total Embedding API Costs (Estimated)}
\begin{tabular}{lrr}
\toprule
Tier & Method & Estimated Cost \\
\midrule
Tier 0 & Exact match & \$0 \\
Tier 1 & Brand embeddings ($\sim$65K strings) & \$0.01 \\
Tier 2 & Singleton embeddings ($\sim$20M strings) & \$2--5 \\
Tier 3 & String distance & \$0 \\
\textbf{Total} & & \textbf{\$2--5} \\
\bottomrule
\end{tabular}
\end{table}

Note: Earlier project estimates of \$500--2,000 were based on larger embedding models and outdated pricing. Current \texttt{text-embedding-3-small} pricing (\$0.02/1M tokens) makes full-scale matching highly affordable.

\subsection{Limitations}

\begin{enumerate}
    \item \textbf{Name variation}: Despite embedding-based matching, some legitimate matches may be missed due to substantial name differences (e.g., ``Alphabet Inc.'' vs. ``Google''; parent-subsidiary relationships)

    \item \textbf{Subsidiary relationships}: Our matching does not automatically capture corporate hierarchies. A brand owned by a conglomerate matches to the brand-level PAW record, not necessarily the parent. We provide \texttt{ultimate\_parent\_rcid} for researchers who wish to aggregate to parent level.

    \item \textbf{Private company precision}: Companies without ticker symbols rely entirely on name matching, which may have lower precision than ticker-based matching.

    \item \textbf{Geographic blocking errors}: The MSA blocking strategy assumes local businesses operate within MSA boundaries. Companies headquartered in one MSA with employees recorded in another may cause false negatives.

    \item \textbf{Temporal misalignment}: PAW employment records and Advan POI data may reflect different time periods. Company names change; locations open and close.

    \item \textbf{Singleton match noise}: Tier 2 matching for unbranded POIs is inherently noisier than brand matching. Researchers should consider restricting analysis to branded POIs or applying match quality filters.

    \item \textbf{Non-employer POIs}: Some Advan POIs are not employers (parks, landmarks, residential buildings miscoded as commercial). These will not match to PAW and should not be interpreted as match failures.
\end{enumerate}

\subsection{Replication}

To replicate this entity resolution procedure:

\begin{enumerate}
    \item Obtain access to Advan POI data (commercial license via Dewey Data or direct from Advan)
    \item Obtain access to Politics at Work employment data (restricted access via Harvard Dataverse)
    \item Download NBER CBSA-to-FIPS crosswalk (publicly available)
    \item Configure OpenAI API access for embedding generation
    \item Execute pipeline scripts in order:
    \begin{enumerate}
        \item \texttt{00\_filter\_us\_pois.py}: Filter to US POIs
        \item \texttt{01\_extract\_unique\_entities.py}: Extract unique brands/companies
        \item \texttt{02\_tier0\_ticker\_match.py}: Exact ticker matching
        \item \texttt{03\_tier1\_brand\_embeddings.py}: Brand embedding matching
        \item \texttt{04\_tier2\_singleton\_msa.py}: Singleton MSA-blocked matching (array job)
        \item \texttt{05\_combine\_crosswalks.py}: Combine all tiers
    \end{enumerate}
\end{enumerate}

Scripts are available at: \url{https://github.com/maxikagan/measuring-stakeholder-ideology}


\section{Methodological Appendix: Validation Against Twitter-Based Measures}
\label{sec:appendix_validation}

\subsection{Overview}

This appendix validates our foot traffic-based consumer partisan lean measures against an alternative approach: Twitter-based brand ideology scores from \citet{schoenmueller2023polarized}. The Schoenmueller et al. measure infers brand political positioning from the partisan composition of brands' Twitter followers, providing an independent benchmark derived from entirely different data and methodology.

We compare the two approaches at the brand level, examining correlation, concordance, and systematic differences. The goal is to assess convergent validity: if both measures capture the same underlying construct (consumer political composition), they should be positively correlated despite methodological differences.

\subsection{The Schoenmueller et al. Measure}

\citet{schoenmueller2023polarized} develop a measure of brand political orientation by analyzing the Twitter follower networks of major consumer brands. Their approach proceeds as follows:

\begin{enumerate}
    \item \textbf{Follower Collection}: For each brand's official Twitter account, they collect a sample of followers.

    \item \textbf{Political Classification}: Each follower is classified as liberal or conservative based on whether they follow predominantly liberal or conservative political accounts (politicians, pundits, media outlets).

    \item \textbf{Brand Score Construction}: The brand's political orientation is computed as the share of followers classified as conservative, yielding a score from 0 (entirely liberal followers) to 1 (entirely conservative followers).
\end{enumerate}

The resulting dataset covers 1,289 brands across multiple consumer categories, including retail, restaurants, consumer packaged goods, and services. The measure has been validated against consumer surveys and predicts partisan differences in brand attitudes and purchase intentions.

\subsection{Linking the Datasets}

We link Schoenmueller et al. brand scores to our Advan-based measures using brand name matching. The linking process involves:

\begin{enumerate}
    \item \textbf{Semantic Embedding}: We embed both Schoenmueller and Advan brand names using OpenAI's \texttt{text-embedding-3-large} model (1536 dimensions), then identify candidate matches via cosine similarity.

    \item \textbf{String Distance}: We compute Jaro-Winkler similarity for all candidate pairs, which effectively handles common brand name variations (e.g., ``McDonalds'' vs. ``McDonald's'', ``7eleven'' vs. ``7-Eleven'').

    \item \textbf{Candidate Selection}: We retain pairs with either (a) Jaro-Winkler $\geq$ 0.85, or (b) cosine similarity $\geq$ 0.85 and rank 1, yielding 1,036 candidate pairs.

    \item \textbf{Manual Verification}: Each candidate pair was manually classified as a true match or false positive, accounting for brand variants (e.g., ``Foot Locker'' and ``Kids Foot Locker'' both match to ``Footlocker'').
\end{enumerate}

\begin{table}[htbp]
\centering
\caption{Brand Matching: Schoenmueller et al. to Advan}
\label{tab:brand_matching}
\begin{threeparttable}
\begin{tabular}{lc}
\toprule
& Count \\
\midrule
Schoenmueller et al. brands & 1,289 \\
Candidate pairs generated & 1,036 \\
Manually verified as TRUE match & 662 \\
Manually verified as FALSE match & 374 \\
\midrule
Unique Schoenmueller brands matched & 416 \\
Final validation sample (with foot traffic data) & \textbf{283} \\
\bottomrule
\end{tabular}
\begin{tablenotes}[flushleft]
\small
\item \textit{Notes}: Matching between Schoenmueller et al. (2023) Twitter-based brand scores and Advan branded POI data. Candidate pairs selected via cosine similarity of semantic embeddings and Jaro-Winkler string distance. Each pair manually verified. Final sample excludes brands without foot traffic-based partisan lean estimates.
\end{tablenotes}
\end{threeparttable}
\end{table}

\subsection{Correlation Analysis}

Table~\ref{tab:validation_correlation} presents correlation coefficients between our foot traffic-based consumer partisan lean and the Schoenmueller et al. Twitter-based measure.

\begin{table}[htbp]
\centering
\caption{Correlation: Foot Traffic vs. Twitter-Based Consumer Partisan Lean}
\label{tab:validation_correlation}
\begin{threeparttable}
\begin{tabular}{lccc}
\toprule
Sample & N & Pearson $r$ & Spearman $\rho$ \\
\midrule
All matched brands & 283 & 0.27*** & 0.40*** \\
\midrule
\textit{By Brand Geographic Coverage} & & & \\
National (31+ states) & 120 & 0.32*** & 0.44*** \\
Regional (6--30 states) & 113 & 0.21** & 0.33*** \\
Local (1--5 states) & 25 & 0.18 & 0.25 \\
\bottomrule
\end{tabular}
\begin{tablenotes}[flushleft]
\small
\item \textit{Notes}: Correlations between foot traffic-based visitor Republican lean (our measure, 2020 election data) and Twitter follower-based conservative share (Schoenmueller et al. 2023). Pearson $r$ is linear correlation; Spearman $\rho$ is rank correlation. *** $p < 0.001$, ** $p < 0.01$, * $p < 0.05$. National brands show stronger correlations, consistent with larger sample sizes and more precise estimates.
\end{tablenotes}
\end{threeparttable}
\end{table}

\subsection{Scatter Plot Visualization}

Figure~\ref{fig:validation_scatter} presents a scatter plot comparing the two measures at the brand level.

\begin{figure}[htbp]
\centering
\includegraphics[width=0.85\textwidth]{figures/validation_scatter.pdf}
\caption{Validation: Foot Traffic-Based vs. Twitter-Based Consumer Partisan Lean}
\label{fig:validation_scatter}
\begin{minipage}{0.9\textwidth}
\footnotesize
\textit{Notes}: Each observation is a brand matched across both datasets (N = 283). X-axis: Conservative share of Twitter followers from Schoenmueller et al. (2023). Y-axis: Visit-weighted average Republican two-party vote share of visitor home CBGs (our measure). Dashed line indicates 45-degree perfect agreement. Red solid line is OLS best fit (slope = 0.13, SE = 0.03). Point size proportional to log total visits. Pearson $r$ = 0.27, $p < 0.001$.
\end{minipage}
\end{figure}

\subsection{Sources of Divergence}

The two measures should correlate positively but need not agree perfectly. Several factors may cause divergence:

\subsubsection{Construct Differences}

The measures capture related but distinct constructs:

\begin{itemize}
    \item \textbf{Our measure}: Political composition of \textit{actual visitors} to physical locations, weighted by visit volume.

    \item \textbf{Schoenmueller et al.}: Political composition of \textit{Twitter followers} of brand accounts, which reflects social media engagement rather than in-store patronage.
\end{itemize}

Twitter followers may skew younger, more politically engaged, and more urban than the general customer base. Brands with strong social media presence may attract followers who differ from typical in-store customers.

\subsubsection{Geographic Coverage}

Our measure captures the full geographic footprint of each brand, including rural locations where Twitter penetration may be lower. The Schoenmueller et al. measure may disproportionately reflect urban/suburban customers who are more active on Twitter.

For brands with differential penetration across political geographies---e.g., a chain with both urban and rural locations---the two measures may diverge.

\subsubsection{Temporal Differences}

The Schoenmueller et al. data were collected at a specific point in time (circa 2019--2020), while our foot traffic data span 2019--2024. If brand political associations have shifted over time, this could introduce divergence.

\subsubsection{Measurement Error}

Both measures contain error:

\begin{itemize}
    \item Our measure relies on ecological inference from CBG-level voting to individual visitor preferences.

    \item The Schoenmueller et al. measure relies on inferring follower politics from their following patterns, which may misclassify moderate or cross-partisan individuals.
\end{itemize}

Classical measurement error in both measures would attenuate the observed correlation toward zero.

\subsection{Complete Brand Comparison}

Table~\ref{tab:full_brand_comparison} presents all matched brands, sorted by number of locations (descending). This ordering prioritizes major national brands where both measures have the most data and highest precision.

\begin{longtable}{lcccc}
\caption{Complete Brand-Level Comparison: Foot Traffic vs. Twitter-Based Measures}
\label{tab:full_brand_comparison} \\
\toprule
Brand & Locations & Foot Traffic & Twitter & Diff \\
\midrule
\endfirsthead
\multicolumn{5}{l}{\textit{Table~\ref{tab:full_brand_comparison} continued}} \\
\toprule
Brand & Locations & Foot Traffic & Twitter & Diff \\
\midrule
\endhead
\midrule
\multicolumn{5}{r}{\textit{Continued on next page}} \\
\endfoot
\bottomrule
\multicolumn{5}{p{0.95\textwidth}}{\footnotesize\textit{Notes}: All matched brands sorted by number of U.S. locations (descending). Foot Traffic = visit-weighted average Republican two-party vote share of visitor home CBGs. Twitter = conservative share of brand's Twitter followers from \citet{schoenmueller2023polarized}. Diff = Foot Traffic $-$ Twitter; positive values indicate our measure is more Republican.} \\
\endlastfoot
Westernunion & 76,673 & 0.52 & 0.53 & $-0.02$ \\
Duh\_Progressive & 21,821 & 0.49 & 0.88 & $-0.39$ \\
Walmart & 21,097 & 0.55 & 0.52 & $+$0.03 \\
Subway & 20,467 & 0.51 & 0.51 & $+$0.00 \\
Dollargeneral & 18,314 & 0.59 & 0.52 & $+$0.07 \\
Walgreens & 17,555 & 0.47 & 0.46 & $+$0.01 \\
Starbucks & 15,576 & 0.47 & 0.44 & $+$0.03 \\
Mcdonalds & 13,635 & 0.50 & 0.52 & $-0.01$ \\
7eleven & 12,946 & 0.46 & 0.45 & $+$0.00 \\
Dunkindonuts & 9,140 & 0.45 & 0.49 & $-0.04$ \\
Sherwinwilliams & 8,546 & 0.50 & 0.55 & $-0.05$ \\
Hrblock & 8,311 & 0.48 & 0.47 & $+$0.01 \\
Dollartree & 7,864 & 0.50 & 0.47 & $+$0.03 \\
Tacobell & 7,718 & 0.51 & 0.50 & $+$0.01 \\
Burgerking & 7,049 & 0.49 & 0.51 & $-0.02$ \\
Chevron & 6,835 & 0.50 & 0.61 & $-0.11$ \\
Rotary & 6,711 & 0.49 & 0.41 & $+$0.08 \\
Pizzahut & 6,670 & 0.49 & 0.52 & $-0.03$ \\
Exxonmobil & 6,633 & 0.53 & 0.62 & $-0.10$ \\
Tmobile & 6,569 & 0.46 & 0.47 & $-0.02$ \\
Dominos & 6,341 & 0.51 & 0.53 & $-0.01$ \\
Wendys & 5,961 & 0.50 & 0.51 & $-0.00$ \\
Att & 5,305 & 0.48 & 0.50 & $-0.03$ \\
Boostmobile & 4,782 & 0.42 & 0.48 & $-0.07$ \\
Chase & 4,751 & 0.42 & 0.49 & $-0.07$ \\
Wellsfargo & 4,509 & 0.45 & 0.54 & $-0.09$ \\
Riteaid & 4,493 & 0.44 & 0.45 & $-0.01$ \\
Acehardware & 4,347 & 0.54 & 0.61 & $-0.07$ \\
Kfc & 3,895 & 0.51 & 0.53 & $-0.02$ \\
Enterprise & 3,783 & 0.48 & 0.49 & $-0.01$ \\
Bankofamerica & 3,759 & 0.41 & 0.51 & $-0.10$ \\
Kroger & 3,525 & 0.51 & 0.51 & $+$0.00 \\
Arbys & 3,408 & 0.55 & 0.53 & $+$0.01 \\
Citgo & 3,304 & 0.50 & 0.72 & $-0.22$ \\
Speedway & 3,293 & 0.51 & 0.59 & $-0.08$ \\
Remax & 3,095 & 0.49 & 0.63 & $-0.14$ \\
Papajohns & 3,037 & 0.51 & 0.58 & $-0.07$ \\
Gamestop & 2,941 & 0.47 & 0.54 & $-0.07$ \\
Holidayinn & 2,875 & 0.51 & 0.49 & $+$0.02 \\
Popeyes & 2,851 & 0.45 & 0.44 & $+$0.01 \\
Chickfila & 2,799 & 0.50 & 0.70 & $-0.21$ \\
Fedex & 2,760 & 0.46 & 0.52 & $-0.06$ \\
Publix & 2,577 & 0.52 & 0.58 & $-0.06$ \\
Ymca & 2,463 & 0.50 & 0.40 & $+$0.10 \\
Hertz & 2,415 & 0.47 & 0.48 & $-0.01$ \\
Sears & 2,356 & 0.50 & 0.49 & $+$0.00 \\
Bestwestern & 2,329 & 0.51 & 0.48 & $+$0.03 \\
Aldiusa & 2,286 & 0.50 & 0.45 & $+$0.05 \\
Phillips66gas & 2,244 & 0.59 & 0.67 & $-0.08$ \\
Panerabread & 2,155 & 0.46 & 0.45 & $+$0.01 \\
\midrule
\multicolumn{5}{l}{\textit{...233 additional brands with fewer than 2,000 locations omitted...}} \\
\end{longtable}

\subsection{Conclusion}

The validation analysis reveals moderate convergent validity between our foot traffic-based measure and the Schoenmueller et al. Twitter-based measure. The positive correlation ($r = 0.27$, $p < 0.001$) indicates that both approaches capture meaningful variation in consumer political composition, despite using entirely different data sources and methodologies. The Spearman rank correlation ($\rho = 0.40$) suggests stronger agreement in relative orderings than absolute values.

The modest linear correlation reflects systematic differences between the measures rather than noise. Twitter followers self-select based on brand affinity and political identity, while physical visitors are constrained by geography and practical needs. Brands with extreme Twitter followings (e.g., Trump properties with 92\% conservative followers) show the largest divergence from foot traffic patterns, consistent with ideological self-selection on social media. Indeed, the correlation between a brand's Twitter-based extremity and divergence from foot traffic is $r = 0.72$, indicating that Twitter captures performative political consumption while foot traffic captures routine commercial behavior.

The measures are not interchangeable. Each has strengths and limitations: our measure captures actual physical visits across the full geographic footprint; the Twitter measure may better capture social media-engaged consumers and brand perception. The choice between measures should depend on the research question.

For our purposes---examining stakeholder pluralism using comprehensive coverage of commercial activity---the foot traffic approach offers advantages: broader coverage (millions of establishments vs. thousands of brands), granular geographic resolution, and linkage to employee data. The validation against an independent external measure increases confidence that our estimates capture meaningful political composition variation.



\end{document}
